\subsection{Before You Begin}
We are working to make PlanetMath into a consistent, correct, and
comprehensive, free mathematical resource.

When we say ``free'', we are refering to \emph{freedom}, not price.
The PlanetMath encyclopedia is released under the Creative Commons
Share Alike license.  Using this license for your work means that
other people around the world will be able to copy and modify your
contributions in ways you hadn't necessarily imagined.

In order to be an effective PlanetMath contributor, you should be
aware of the responsibilities you take on when contributing.
One important responsibility is to only contribute your own writing,
or other texts that you know you have a legal right to add. This is
discussed in the document is about copyright, and it is important that
you understand the issues presented there. The other responsibilities
have to do with maintaining your entries and will be discussed in this
document. 

One thing to bear in mind is that while we want our work to be
consistent and correct, it is not expected that you get things perfect
the first time. On the contrary, writing a correct and complete entry
is an iterative process. We caution you against expecting to be
precisely and exhaustively correct on your first (or second, or third)
attempt! You should not be afraid of receiving corrections and
suggestions from others, and in fact you should expect them.

Do not expect to retain ``ownership'' of your entries if you will not
have time to maintain them. There are plenty of people who will be
willing to adopt abandoned entries. If you do not respond to
corrections in a timely fashion, your entries will eventually be
considered to have been abandoned, and they can then be adopted by
someone else.

Part of the benefit of PlanetMath is the collaborative nature of the
project: math enthusiasts from all over the world want to share what
they know, and learn through sharing and discussion. If you do not
expect to learn and think you know it all beforehand, PlanetMath is
probably not for you.

\subsection{Metadata}
\textit{Metadata} is a word meaning ``data about data''. For our
purposes, this means information about the main (\LaTeX) content of
your entry. Much of this document is about metadata for PlanetMath
entries. This includes titles, synonyms, defines (sub-definitions),
type, keywords, and classification.

To make your entry properly fit in with the rest of PlanetMath, it is
important that you understand how to best write its metadata. This is
not complicated, but perhaps not obvious to beginners, so read on to
see how.

\subsubsection{Classification}
Since the purpose of our collection is to provide coverage of all areas
of mathematics, some sort of organization is needed.  This is provided
the classification system --- once entries are classified, they can be
 browsed by subject, through a subject classification hierarchy. For 
this to work properly, we rely on owners to classify their entries 
appropriately.

Currently PlanetMath only supports MSC, the AMS Mathematical Subject 
Classification scheme. MSC is very widely used and is more or less 
exhaustive over known mathematics -- you probably will never run into 
an entry that can't be classified with MSC (at least to one level in the
 hierarchy.)

The MSC takes some getting used to and is not the sort of thing one
is likely to recall off the top of one's head . Thus, in order to make things 
easier, we have set up a local copy of the MSC which is hierarchically 
browseable and searchable, and is accessible from the menu.

\subsubsection{Subjects or sub-disciplines in titles}
For homonyms (ambiguous terms like ``algebra'', ``domain'', or
``complex''), it often seems appropriate to append a parenthesized
``subject hint''. For example, one might think the smart thing to do
is name an entry ``diagonalization (Cantor)'' to avoid conflation with
the linear algebra sense of ``diagonalization''. 

Since subject classification takes care of disambiguating most such conflicts 
(e.g.``diagonalization (linear algebra)'' would be classified as algebra
 whilst ``diagonalization (Cantor)'' would be classified as set theory)
this is usually not necessary, but may nevertheless be convenient.
For instance, it might be done this with the encyclopedia index listing 
in mind (that is, it might be nice to see``diagonalization (linear algebra)''
in the index.)  In such cases, adding a parenthesized ``subject hint'' to 
your title is acceptable provided the plain title is at least given as a 
synonym (and the entry is still properly classified.) 

\subsubsection{Types}
There are a number of types which are available for describing what
the mathematical form of your entry is. These are:
\begin{itemize}
\item Definition
\item Theorem
\item Conjecture
\item Axiom
\item Topic
\item Biography
\item Algorithm
\item Data structure
\item \textit{Proof}
\item \textit{Result}
\item \textit{Example}
\item \textit{Derivation}
\item \textit{Corollary}
\item \textit{Application}
\end{itemize}
The entries in italics are meant to be attached to other entries. They
do not show up in the encyclopedia index, so placing an entry under
one of these categories has important practical as well as
philosophical ramifications.

An example of why types matter: a definition should not have proof,
since definitions have no truth value -- but it may have a
derivation. Hence, you cannot attach a proof to a definition
(actually, you can, but it is discouraged.)

A theorem may have a proof, and in fact it should be provided for a
full acceptance of the theorem as a theorem. Hence, PlanetMath makes
it easy for a proof to be attached to a theorem (and only a theorem.)
As before, you actually can attach a theorem to anything, but doing so
is less convenient and is discouraged.

Examples are meant to be used everywhere. They allow some of the load
to be taken off the primary entry author, by allowing the community of
users to pedagogically enrich existing entries.

The ``Conjecture'' type might be a little confusing to some. In terms
of how the system treats conjectures, they are the same as
theorems. That is, they are meant to have proofs attached to them, as
well as results or corollaries. This makes sense, since a conjecture
is basically treated as a yet-unproven theorem. However, when one
looks at a topic like the Taniyama-Shimura conjecture (which has now
been proven), its hard to decide which type is more
appropriate. Proven conjectures may still be better left as
conjectures by convention. The opposite situation is a conjecture
which is considered a theorem before its time -- like Fermat's last
theorem. Yet another situation might occur when it turns out a
conjecture (like the Continuum Hypothesis) is unprovable (can only be
used as an axiom). There is no single answer for these situations, you
simply must take into account practical considerations (for instance,
that conjectures won't show up in ``unproven theorems'') and
convention on a case-by-case basis. Don't worry too much, however,
about picking the absolute best type the first time around in such an
ambiguous situation.

\subsubsection{Synonyms and Definitions}
PlanetMath provides a ``synonym'' field for entries. The obvious
things to put in here are alternate names for your entry. The
not-so-obvious thing is that you should also be thinking of linking
when you do this. That is, you should list all aliases for your entry
that someone else might invoke in other entries, to faciliate
automatic linking.

You do not, however, need to make extra synonyms for variants of
pluralization, possessiveness, or transmogrifying ``Blah, proof of''
into ``proof of Blah''. \emph{These are done automatically by
PlanetMath.}

\paragraph{Examples.}
{\bf title:} \emph{Euler's totient function}, {\bf synonym:}
\emph{Euler totient function}. Wrong -- the synonym is just the
nonpossessive of the title; leave that for the software to handle!

{\bf title:} \emph{Cauchy-Schwarz inequality}, {\bf synonym:} \emph{Kantorovich's inequality}.
Correct -- both names are used to refer to the same thing.

{\bf title:} \emph{monotonic}, {\bf synonyms:} \emph{monotone,
monotonically}. Correct -- we want all occurrences of ``monotonic'',
``monotone'' and ``monotonically'' to link to the same object.

{\bf title:} \emph{vector valued function}, {\bf synonyms:} \emph{vector-valued,
vector-valued function, vector valued}. Correct -- we have to take care
of variants in hyphenation as well as the particular set of words.

In addition, there is a ``defines'' field which provides for
``sub-definitions'' of your entry. This facility allows you to define
some set of new concepts all at once in a single entry (for example,
it might be better to define ``edge'' and ``vertex'' within a
``graph'' entry, instead of separately). Each of these
``sub-definition'' handles will be treated appropriate by PlanetMath's
automatic linking when they are invoked in other entries (they will
get hyperlinked, whereas multiple synonyms to the same entry will
not.)

\paragraph{Examples.}
\begin{itemize}
\item An entry for ``graph'' may also define ``vertices'' and
``edges'' and hence have ``vertices, edges'' as the ``defines''
field.
\item An entry for ``Zermelo-Fraenkel axioms'' may also list as
sub-definitions each individual axiom, i.e. defines=``axiom of empty
set, axiom of infinity, ...''
\item An entry for ``Taniyama-Shimura conjecture'' might also have
synonyms ``Taniyama-Shimura-Weil conjecture'', ``Taniyama-Shimura
theorem'', and ``Taniyama-Shimura-Weil theorem'', and hence list
these as synonyms. These would not be listed in the ``defines''
field -- if two of these terms are invoked from the same entry, they
should not both be linked, which will be the case if they are listed
as synonyms.
\end{itemize}
It is important to note that there is no general rule for the exact
``granularity'' of entries -- things that ``stand on their own''
should be their own entry, but this is hardly a rigorous metric
(however, if you choose to combine things that could be separate
entries, you should provide a ``defines'' list for sub-definitions.)
Use your best judgement, and you'll probably hear from others if
there's disagreement.

\subsection{Corrections}
What you should file corrections for:
\begin{description}
\item[Mathematical Errors] These may be as simple as a typo or as
serious as a completely erroneous proof.
\item[Typographical errors and grammatical errors] PlanetMath should
be as ``professional'' as any published book or encyclopedia (in
fact, there is little excuse for the quality of PlanetMath not to
exceed fixed media for the set of ``stable'' entries.) As such,
please point out even the smallest of mistakes, if they truly are
mistakes.
\item[Comprehensiveness] If more can be added, it probably should
be. This includes showing relatedness to other branches of
mathematics, and possibly applications. It includes alternate
derivations, additional results and properties, and different
methods of visualization or approaches to explanation. You don't
have to write a book -- that would of course defeat the purpose of
an encyclopedia. But the idea is to mention all of the important
insights so that the reader knows what to look for if they'd like to
study the idea in more detail.
\item[Comprehensibility] Formal and concise statements tend to be
useful for reference purposes, but they are not very useful for
learning what one does not already know. More extensive
explanations, visualizations, and examples are very powerful tools
for teaching, and they should play a large part in nearly all
entries.
\item[Alternative conventions] This is a tough one for most. Often
times there are conventions which vary from country to country,
region to region, school to school, or even class to class. Think of
PlanetMath in a global context when you write and critique entries,
and it should become apparent that probably most alternatives should
at least be mentioned, before a particular choice is made for usage.
\item[Interconnectedness] By this we mean provisions for making
PlanetMath as interlinked as possible. This includes tweaking
mentions of concepts so that they trigger linking to a PlanetMath
entry, or conversely, adding synonyms to entries or tweaking titles
to conform with the way they are mentioned in entries. It includes
adding explicit ``related'' (See also) links to other things in the
encyclopedia when they should be there. Also important is reporting
to an object owner when a link goes to the ``wrong'' entry, or there
is a link where there should not be, and reporting the lack of a
subject classification (which serves as a hint to automatic
linking).
\end{description}
Likewise, you should expect to receive corrections when your entries
are lacking in any of these areas.

Corrections don't always go smoothly. Often you feel a correction was
justified, but the author rejects it. The first thing to do in this
situation is find out if there was a misunderstanding: you can post
messages to the correction and discuss it. You can try filing another
correction wording things differently. When it becomes clear the
author is not going to do things your way, we suggest the approach
from the next section. \textbf{Under no circumstances will the staff
of PlanetMath mediate disagreements about corrections.}

\subsection{Alternate Entries}
You should always run a search before writing an entry to see if
someone else has already covered the same material. However, even if
the ideas have already been discussed, there may still be reason for
you to write an alternate entry. Alternate entries are justified when:
\begin{itemize}
\item you have a radically different treatment of the subject. This
could be another educational level (as in introductory
vs. advanced), or another method (as in a proof, which can have tens
of alternatives).
\item the author of the entry is discarding corrections. In this case
the object will not eventually be orphaned for pending corrections,
so you cannot force modifications to it (do \textbf{not} complain to
the staff in this case; we won't force the other author to do
anything).
\end{itemize}
We would prefer uniformity and cohesion on PlanetMath, but there is a
natural limit to how far this can be stretched with so many different
minds. The lack of scarcity (i.e. limited space) on PlanetMath also
gives it an advantage over traditional media, allowing us to avoid
standardization and provide extra value in yet another way.

\subsection{Questions, Answers, Problems, and Solutions}
