While mathematical concepts can not be owned, their expression is
subject to the strictest protection under copyright law. Furthermore,
one cannot convey mathematical information without expressing it
somehow. There is much more to ``expression'' than simply an
author's choice of words and, accordingly, direct copying is only one
of many sorts of copyright infringements.

Below, we present some guidelines for writing entries that may help
you avoid exposing yourself or PlanetMath.org to legal problems.
While, in producing this document. we have taken care to check that
the facts presented here are correct, this document is intended
only for informational purposes, not as a definitive guide or as legal
advice.  PlanetMath offers no guarantees that following the suggestions
offered here will suffice to prevent infringement.  For authoritative
guidance, the reader is urged to refer the legal literature and seek
professional advice.

\begin{itemize}
\item Bear in mind that a text does not need to have a copyright
notice attached to it in order to be copyrighted. Moreover,
copyright protection has nothing to do with whether the work is
published or unpublished, whether a work is still in print, or
whether the publisher charges for copies of the work. In fact, the
simple act of writing something down automatically confers copyright
protection to the author. In particular, this means that class
notes and handouts, webpages, and newsgroup postings are all legally
protected.
\item If you see the exposition of a certain mathematical topic on
someone's homepage and think it would make a great addition to
PlanetMath, you should do nothing unless you can first obtain the
copyright holder's permission, in writing, to publish a copy of the
work on PlanetMath. Likewise, you may not post a copy of notes that
were handed out in a class, or even notes that you took on a spoken
lecture, without first obtaining written permission. Asking for
permission is also an opportunity to tell others about PlanetMath
and free math. However, if you do not receive the author's permission
to publish, you cannot post the work! If you do
receive the copyright holder's permission to use the work,
\emph{include their permissions statement as an attachment.}
\item When dealing with published material, especially if it is still in
print, keep in mind that authors sign contracts with their publishers
which typically restrict the author's rights. Therefore, even if the
author of a book or an article in a journal gives you permission to
use his work, it may be and likely will be necessary to also obtain the
publisher's permission.
\item Copying from works released under the Creative Commons 
share-alike and attribution licenses is fine, but it requires that you
follow the conditions laid out in the license such as licensing your
work under the same license or attributing the author of the
original work. 
\item Copying from public domain works is also fine. Mathematical
works that are in the public domain are mostly those whose
copyrights have expired, but also works that have been transfered to
the public domain by their authors, as well publications of the US
government. As a rule of thumb, works published in the ninteteenth
century and earlier are in the public domain, but unless you can
find proof to the contrary, twentieth century works are likely to be
off-limits. It is of course wise to give a citation so that others
can easily check the assignment (or expiry) of copyright for
themselves -- and perhaps also find additional useful material from
the same source.
\footnote{When dealing with older works, keep the following points in
mind: (a) The law on when copyright expires is somewhat different
for unpublished works, so these need to be treated as a special
case; (b) Before World War II, English was not the dominant language
of the mathematical community. Therefore, older works are more
likely than contemporary works to appear in a language other than
English. Since translation is a creative act, translations are
protected by copyright, even if the work that was translated is in
the public domain. Thus, if you quote at length from an older work
written in a foreign language, you should either do the translation
yourself, or else find a translation which is also in the public
domain (or FDL'ed).}
\item As a rule of thumb, if you cannot provide at least a sketch of a
given topic without referring to a source, you are probably not yet
qualified to write an entry about that topic. Not only is this
policy prudent from the legal standpoint, it also makes sense from the
point of view of mathematical content. If you rely too heavily on a
given source, you run the risk of perpetuating whatever mistakes and
oversights may be present there. Furthermore, unless you have a
fairly deep understanding of a given topic, you might misunderstand
another author's use of a particular technical term, or forget to
state assumptions which this other author stated in an earlier
chapter. A document written from your own understanding will be much
more useful than one that purports to present facts that you
yourself do not understand. You needn't be a world expert to write
a useful entry -- simply trying to state your own questions clearly
will be much more helpful to everyone involved than it would be for
you to try to mimic someone else's exposition.
\item Cite all sources, including any web pages, lectures, or personal
communications that have informed your work. If possible, summarize
the relationship your article bears to the source or sources you
used. Which parts of the article derive from which sources? Which
parts are original?
\item Keep in mind the fact that, as the copyright office says,
``Acknowledging the source of the copyrighted material does not
substitute for obtaining permission.'' To be sure, documenting the
process you used while writing an article, and which sources you
looked at, could help prove that you did not infringe on anyone
else's copyright -- but whether or not you cite a particular work is
not a factor in determining whether your work infringes on that
work's copyright.
\item Embellish your articles with examples, illustrations, proofs,
and other extensions either of your own devising or drawn, bit by
bit, from a variety of sources -- make your exposition truly your
own. No one part of your article should be too close to anything
drawn from any one source. In addition, neither the overall
structure of your article nor any part of its structure should be
too close to the structure or any non-trivial part of the structure of
any one source.
\item Bear in mind that the particular choice of words that an author
uses is not the only thing that copyright protects: copyright
protects expression in general, and even the particular selection of
facts or ideas that an author chooses to talk about is protected.
\item However, copyright does not protect individual facts, ideas,
concepts -- so write about all these things! But always do it in
your own words, and do not rely too heavily on one source. Even
something as simple as a theorem statement or a non-trivial equation
should be put in your own words, and expressed in a way that is
consistent with other usage on PlanetMath.
\end{itemize}

PlanetMath staff will \emph{not} protect entry authors from the
consequences of any copyright infringement, and rather will do
everything in their power to protect the site from the irresponsible
(though perhaps well-intentioned) actions of persons who seek to
contribute things they do not have the right to.

When it comes to deciding whether some particular use of a copyrighted
source is permissible, the distinction betwen a \emph{derivative work}
and \emph{transformative use} needs to be kept in mind. While there
is no space here to go into details and study examples, at least a
brief description should suffice to make the reader aware of the legal
principles that are in play here, and the fact that there is an
important distinction between the two kinds of use.

A \emph{derivative work} is one whose content has been derived from an
already existing work. Examples include translations, abrigements, or
adaptations. Even though a derivative work can contain a substantial
changes or additions of new material and other original contributions,
a derivative work cannot be prepared without the permission of the
owner of the copyright of the original work on which it is based. (In
fact, minor changes and additions to an existing work do not even
qualify as derivative work, but rather as outright copying.) The 
Creative Commons Share-Alike license gives users permission to publish 
derivative works, so long as the derivative works are released under 
the terms of the same license. In general, copyrighted works which are
not released under that or some similar free license (like th GNU FDL)
offer no such permission to their users.

In contrast, \emph{transformative use} of copyrighted material
consists of putting the material to a different use or function than
that originally intended by the author of the original work. This is
permitted as a fair use of copyrighted material but one needs to be
careful not to take any more of the material than is necessary for
this new purpose. In deciding whether a certain usage is suitably
``transformative'', consideration is given to whether or not the new
work affects the marketability of the original work, or whether it 
satisfies a purpose for which the original work was designed.

One needs to remember that in deciding a copyright infringement case,
courts will consider how much material may have been used without
permission. Thus, it may be OK to have a single short entry that is
rather close to the small section of an original work from which it
derives; however, it is a more serious matter if a whole series of
short entries are all based on the same source.
