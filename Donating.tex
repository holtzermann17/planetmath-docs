
PlanetMath needs your help! While we have been a great success in producing
mathematical content through the volunteer work of the community, our resources
have fallen distressingly short of needs in terms of delivery and development.

In specific, we'd like to hire people to various extents to:

\begin{itemize}
\item perform server administration
\item fix bugs and make small updates to the collaborative system (Noosphere)
\item undertake more extensive software development
\item manage the nonprofit org (maintenance, governance of the community,
      development, support-seeking)
\end{itemize}

Right now, as much as any of these things are done, they are done in what
little volunteer time a few people can spare and are therefore lacking greatly
in consistency and scale. Currently we pull in about \$5000/yr from ads and
donations; we figure we need about \$40,000/yr to really start hiring people (at
first, either a developer/admin or part time of each). Our near-term target for
public donations, however, is \$10,000/yr, and we are about a third of the way
there.

We need your help to improve quality of service of PlanetMath and continue to
help the software and community grow and evolve. We hope that, if you feel
PlanetMath is a great resource, you will consider one of the following forms of
assistance.

Note: PlanetMath is a US-based nonprofit corporation, "PlanetMath.org, Ltd.",
incorporated in Alexandria, Virginia. It is a 501(c)3 tax-exempt public
charity, so donations from US residents are tax-deductable!

Liquid cash is probably the best way to help us out, because we can turn it
into most anything else we need to improve the delivery and development of
PlanetMath. We are open to large-scale grants from nonprofit and government
organizations or private corporations, private and public sponsorship
arrangements, and private donations from individuals. Though we don't have a
formal monetary-base membership arrangement for the PlanetMath.org, Ltd.
corporation (for now), we suggest a standard yearly donation of \$20US-\$100US
for individuals who are able and use PlanetMath regularly or just want to
support the effort.

You can get money to us as follows:

\begin{itemize}
\item Use our PayPal "donate" button.
\item Send a check to:
      PlanetMath Donations
      C/O Bonnie Rabichow
      4336 Birchlake Ct.
      Alexandria, VA 22309
      USA

\item Get matching donations. Many employers will match donations made by their
      employees to registered tax-exempt public charities. This is a very easy
      way to increase the power of your donations. By how much? Before we were
      tax-exempt and employers couldn't match donations, a dollar donated was a
      dollar towards PlanetMath. Now, since you can write off your donation and
      get your employer will match it, every (approx.) 2/3 dollar is two
      dollars for PlanetMath. In other words, each real dollar you contribute
      is three for PlanetMath (given matching and standard tax assumptions)!
      Find out if your employer has a program like this, and give the handler
      your proof of contribution and our tax identification number 42-1605465
      when you make a donation.
\item Buy goodies from our CafePress Store.
      This also has the added benefit of spreading the word about PlanetMath to
      all who see your PlanetMath gear (however, keep in mind we only get a
      small slice of the sale price).
\item Grants or sponsorships - Please contact us at the email address at the
      bottom, or send a letter to the snail mail address above. You can also
      call us vox at 404-712-2810.
      We think that this type of support has the potential to enable more
      forward-thinking development of PlanetMath/Noosphere and collaborative
      digital libraries in general. If you are interested in funding research
      and development in this area, please get in touch with us.
\end{itemize}

Donations of facilities or hardware may be useful to us, depending on the
specifics of the situation. Small-scale hardware, such as spare, stand-alone
home or home-office computers, will probably not be useful to us. However, we
would eagerly consider grants of rack-mountable, mid- to large-scale
multiprocessing or clustering systems.

Volunteering your time and expertise is a great way to help out. We are looking
for people who can code and want to help fix bugs and optimize PlanetMath and
the Noosphere system (as well as to extend them). We chiefly use Perl at the
moment, but we can usually make use of good coders who don't work in that
language. We are also interested in ideas for improving PlanetMath's
architecture to make it more elegant, efficient, maintainable, extensible, and
scalable.

We encourage all prospective volunteers to join in the activities on our
planning and coordination wiki, AsteroidMeta. Here are the PlanetMath and
Noosphere sections.

Please send inquiries about any of the above methods of assistance to
feedback@planetmath.org. Volunteers should check out the collaboration wiki
mentioned immediately above. Thanks in advance! -- Aaron Krowne
President, PlanetMath.org, Ltd.
