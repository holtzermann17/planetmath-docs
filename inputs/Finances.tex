
\subsubsection*{Financial summary}
Historically our income has come from individual donations, Google ads, and
some cash support as part of the Google Summer of Code. In 2006, one of our
major contributors made use of the Microsoft matching donations program,
pushing our income to a record level (over \$10,000) that year.

\begin{table}
\begin{center}
\begin{tabular}{lllll}
{\bf Year} & {\bf Income} & {\bf Expenses} & {\bf Net} \\ 
2005 & 2836.01 & 1266.54 & 1569.47 \\ 
2006 & 11850.21 & 7486.47 &  4363.74 \\
2007 & 4391.63 &  7263.42 & -2871.79 \\ 
2008 & 9694.00 &  3961.14 &  5732.86 \\
\end{tabular}
\end{center}
\caption{Summary of Income and Expenditures}
\end{table}

\begin{table}
\begin{center}
\begin{tabular}{lll}
{\bf Year} & {\bf Income} & {\bf Expenses} \\
2005 & 25\% from ads & 90\% legal fees \\ 
2006 & 14\% from ads & 37\% for honoraria, 20\% for travel, 26\% subscription system \\
2007 & 40\% from ads &  41\% for contract labor, 37\% travel  \\ 
2008 & 11\% from ads &  38\% for contract labor, and 36\% travel \\
\end{tabular}
\end{center}
\caption{Income from ads; major expenditures}
\end{table}

\subsubsection{Projected Annual Budget}

Our goal is to raise \$50,000 this year. This is very close to the estimated
annual "true cost" of PlanetMath's operations. However, an annual budget of
\$50,000 would do more than just maintain the status quo. With this kind of
funding, our work would begin to thrive. In recent years, we've have money in
our bank account that we don't have a good way to disperse, i.e., we don't have
enough money to offer long-term contracts to skilled, motivated workers who
require a degree of stability in their employment. Although we have a volunteer
effort adding up to about 3/4 time of a staff person, this effort is spread out
over people who are not always available.

\begin{table}
\begin{center}
\begin{tabular}{ll}
\$15600 & non-profit administration, half-time position \\
\$15600 & systems administration, half-time position \\
\$15600 & software development, half-time position \\
\$1200  & a server that doesn't crash \\
\$2000  & travel reimbursements \\
\end{tabular}
\end{center}
\caption{Projected Budget}
\end{table}

Although a \$50,000 budget defines and fills the major spending areas, a
benefits package would help us attract first-rate personnel. Similarly, whether
we begin with an annual budget of \$50K or \$75K, salary increases over time will
be required if we're going to retain these individuals. Accordingly, we would
like to see budgetary growth of \$5K to \$10K per year.

\subsection*{True Costs of Operations}
Since 2006, Google Summer of Code has also sponsored a total of 9 internships
at PlanetMath, an in-kind contribution valued at up to \$45,000. We've also
received facilities support from Virginia Tech since our inception, at an
estimated value of \$100/mo. Direct funding for development of Noosphere from
the IMS, Berkeley CPAM, and Springer in 2006-2009 has come to approximately
\$16,000.

Thus, money or services worth \$64000 on the open market have been spent on
behalf of PlanetMath over the past three years. This comes to roughly \$21,333
dollars per year above our average yearly cash expenditures of \$6770 over the
same period of time.

However, the true cost of operating PlanetMath is considerably more than
\$28,103 per year. The volunteer contributions of time and effort from a wide
range of people make PlanetMath what it is. Assuming out of the group of
volunteers involved with overseeing the day-to-day operations of PlanetMath,
there have consistently been 3 persons contributing 10 hours a week that would
be valued at \$15/hr on the open market, site and non-profit management has
donated about \$23,400 per year.

This puts the true operating cost of PlanetMath as an organization at upwards
of \$51,503 per year.

Further information on PlanetMath finances are available
from our Treasurer, Bonnie Rabichow
(\url{bonsuerab@gmail.com}).

\subsection{The Societal Value of PlanetMath}

User scores are meant to give a relative estimate of the value different users
have contributed to PlanetMath. However, we can also use these scores to
estimate the total value of user contributions. Here are two such estimates
(using data from March 9, 2009):

\begin{table}
\begin{center}
\begin{tabular}{lll}
                                    & {\bf \$0.05 per point} & {\bf \$0.50 per point} \\
Value of a new article              & \$10             & \$100 \\
Highest scoring user's contribution & \$5,145          & \$51,449 \\
Top fifty user's contribution       & \$55,371         & \$553,714 \\
Top one hundred user's contribution & \$62,388         & \$623,884 \\
Approxmiate total user contribution & \$75,000         & \$750,000 \\
\end{tabular}
\end{center}
\caption{Estimated value of user contributions}
\end{table}

Regardless of which estimate is more accurate, the question of how much it
would cost to build the website is, by itself, not a particularly useful
estimate of the value of the site.

For one thing, since PlanetMath is built by volunteers, so we can assume that
each volunteer actually gets something valuable out of contributing, and
replacing their labor power on the open market would remove this "volunteer
surplus". The volunteer surplus is what people would pay to be a contributing
members of PlanetMath. Our suggested minimum membership donation is \$20/year,
although we waive that fee in the case of users who contribute sufficient
content. If we use this amount as our estimate of the average volunteer
surplus, then we'd expect one hundred active users to generate a surplus of
\$2000 per year.

Second, the fact that we get 12,000 to 20,000 web hits per day means that there
are many users who get value out of PlanetMath without contributing content. We
note that if one percent of PlanetMath visitors each day found their experience
worth \$1.25, their small monetary contributions would cover the real cost of
running the site. Similarly, our real costs would be covered if every single
page hit earned \$0.0125 for PlanetMath. If this billing rate per hit reflects
the real value of PlanetMath to non-contributing users, then over the past 5
years, we have delivered over a quarter of a million dollars worth of content
to the mathematical public.

This suggests that the value of PlanetMath in both capital
accumulation and services rendered to date is at least
\$335,000, and perhaps as much as a \$1 mil.

\subsection*{Towards a sustainable business model for PlanetMath}

Several categories of income sources could be drawn on to raise funds
sustainably. 

\subsubsection*{Donations From Community Members}
As noted above, if one percent of PlanetMath visitors each day found their
experience worth \$1.25, their small monetary contributions would cover the real
cost of running the site. Users should be made aware of this and encouraged to
provide \emph{small} donations when they find the site useful. Now that we have a
membership system, we should be doing membership drives and donation drives
more regularly, in particular, so that we can be understood. We need better
practices for running funddrives!

\subsubsection*{Switch from Advertising to Sponsorship}

As noted above, our real costs would be covered if every single page hit earned
\$0.0125 for PlanetMath. If we could make an arrangement with a corporate
sponsor willing to pay \$12.50 CPM or about \$150.00 per day, this would meet our
costs. Even something considerably lower than this would still be an
improvement over Google AdSense.

\subsubsection*{Rendering Services to Outside Entities}
Lately one of our biggest sources of productive income has been contract work
with other organizations (IMS, Berkeley CPAM, Springer) for improvements to our
open source software platform, Noosphere. With a stable staff, we will be in a
better position to be able to provide dedicated services to entities interested
in mathematics research, education, or publishing, or who are interested in
using our software.

\subsubsection*{Selling Copies of the Encyclopedia and Other Derived Works}
A stable staff, along with the roll-out of recent changes to the Noosphere
platform, will help us create a centralized editorial system whose role it
would be to certify certain articles from the PlanetMath encyclopedia for
inclusion in a print encyclopedia. Combining editorial improvements with
existing code that automates a significant portion of the technical editing
needs, we could produce new editions and supplements to our print encyclopedia
at regular intervals. The Encyclopedic Dictionary of Mathematics sells for
about \$100; a typical Springer graduate text sells for about \$50. Our ability
to turn the content of the web-based PlanetMath corpus into a variety of
derivative paper-based products is what stands between us and opening this new
revenue stream.

\subsubsection*{Running a Tutoring Service}
The idea of running a tutoring or question-answering service through PlanetMath
has been circulated many times over the past years. A stable staff and the
right partnerships would help us get such a service of the ground. Receiving
tutoring over the internet is not a new thing, but having the tutoring service
integrated with an open, well-indexed, electronic archive of past answers and
other mathematical content would make PlanetMath an attractive option for
students seeking help.
