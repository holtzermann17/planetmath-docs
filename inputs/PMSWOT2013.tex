\section*{Strengths}

\paragraph{PlanetMath is reasonably well known.}  It continues to be used,
e.g. people are asking reasonably high-level questions.

\paragraph{It will be even more usable with the new software.}  The software
is a field-tested open source mathematics-specific communication
platform.

\paragraph{Our basic technical strengths may help us become a leader in
  mathematics education and research.}  However, in order to get
there, we'll have to build on our technical strength to develop
improved support for specific audiences.  This has implications both
for software development and site policies (particularly editorial and
credit-assigning policies).

\paragraph{PlanetMath integrates features that so far only are being
  delivered by other disparate sites or that don't exist elsewhere.}
Our competitors include Wikipedia, MathOverflow, Boundless.

\paragraph{PlanetMath is a more flexible than the other sites operating in
  the online math space.}  The incentives for using and further
developing PlanetMath differ from what we find in these other systems
in constructive ways.  We're not confined to just be one kind of
system.

\paragraph{PlanetMath isn't evil.}  We want people to contribute, but this
is in a way that serves a broader social mission that most people
would agree is a good and valuable one.

\paragraph{We endeavour to play well with others.}  People will be able to
get a lot of data about PlanetMath over Semantic Web protocols, and
this open data approach is extensible (e.g. if researchers want more
information about how things work on Planetmath, we can share more
data).

\paragraph{We have been consistently working on the project for over a
  decade.}  With experience, we are starting to think more
strategically.  The new more adaptable software platform should help
with that, as will regular attention to re-creating this SWOT
document.

\paragraph{We're gaining experience managing our software development.}  We
have a ticket tracker set up for Planetary, which is a good start for
getting a workflow management system set up that will drive
organizational development.

\paragraph{people involved in the project have backgrounds in AI, knowledge
  management, and relevant fields.}  This will help us build improved
annotation, recommendation, clustering, tools, etc. (E.g. we ought to
be able to make the autolinker link the word ``set'' only when it is
used as a technical noun, not when it's used as a verb).

\section*{Weaknesses}

\paragraph{PlanetMath cannot rest on its laurels.}  Web 2.0 has been around
for some time and most PM features are supported in competitor sites.
Wikidata and SemanticWiki are ahead in terms of Semantic Web
developments. Wikipedia is miles ahead in terms of
internationalization, and that is a feature that takes a dramatic
amount of work to ``do right''.  WriteLaTeX/SpanDeX/ShareLaTeX have
better LaTeX-on-the-cloud setups.  MathOverflow and
math.stackexchange.com sport a generally better forum experience if
you're willing to work with the Q\&A format.  If we're going to be
pioneers, we'll have to set out in some new directions.

\paragraph{The quality of exposition on PlanetMath is quite uneven, some
  articles are sketchy.}  There may be some technical tricks we can do
to help ameliorate these issues, like following links generated by the
autolinker to bring extra information into the page to provide more
complete definitions, building tools to allow people to fork articles,
adding more problems as a method of critiquing the encyclopedia's
content.

\paragraph{There are currently not very many people around to answer
  questions.}  Math is a huge area, but relatively speaking,
PlanetMath is small.  Competition in this area on sites like
MathOverflow and math.stackexchange.com means that we'll need to have
something else good to offer as we build our question-answering
community.  The software for questions and problems is also new, so it
will need to mature.

\paragraph{Our cash flow problems are still not sorted out.}  PlanetMath has
so far been managed mainly as a series of ``student projects''.
Organizational aspects are in more coherent shape than they have been
in the past, and the open source development workflow is going
reasonably well.  In these ways we're at least providing evidence that
we can manage any money that does come in.

\paragraph{The ``anarchistic'' model on PlanetMath has produced some curious
  things (like the content committee) and also a lot of chaos.}  With
a more bit structure in updated by-laws (and corresponding software
implementation), we could probably get more value for our effort.

\section*{Opportunities}

\paragraph{We could have a big impact developing countries where they don't
  have great math libraries.}  But to do this in a focused way, we'll
have to network with the relevant parties.  A Chinese language edition
might be a good idea.  It would also be good talk to Indian and
African mathematical societies -- interesting things are happening
there, and it would be good to get in on it.  Unlike at the MAA, we
may only have to show that what we're doing works at some relatively
basic level, with potential for growth and improvement.

\paragraph{We could make a ``systematic'' survey of our coverage, to build
  an index of what's out there on Wikipedia, Arxiv, MathOverflow.}
This would point the way toward more completeness.  Note that
``coverage'' doesn't just mean having an article on each topic, but
having good exposition too.

\paragraph{We could construe some of our technical weaknesses (missing
  command line integration and email based access) as opportunities to
  serve new sectors.}  In order to fulfil this ``promise'', we'll need
to keep working to maintain and strengthen Planetary development.

\paragraph{The current Semantic Web extensions to Planetary will help us
  build a catalog.}  What we have now starts to provide a model of
what's there.  As time goes by, we may be able to provide a much
richer picture, e.g. getting at the semantics of article quality.

\paragraph{We've centralized the documentation, and we're starting to use a
  Creative Commons style approach to managing docs.}  That is, provide
a clear summary, and then follow with the full technical details for
anyone who's interested.

\paragraph{PlanetMath itself is on more solid footing, so we're ready for
  more outreach.}  We explicitly did not target outreach until we had
a stronger core.  We have a great advisory board made up of people who
are involved with interesting projects, and this is one place to start
our outreach.

\paragraph{As long as we keep research in our workflow, we can learn more as
  we go.} Joe is planning one more study, which we can use as an
opportunity to learn about why people contribute, and
what they perceive as useful.  

\paragraph{The ``books project'' has started on a prototype scale.}  This
can be converted into a huge strength later.  An influx of new content
will fit quite nicely with the RDF model and the improved NNexus
linker.  We can look around for other appropriately licensed content
and start adding it as ``Problems'' and ``Collections'', etc.

\paragraph{We've established a good relationship with KWARC.} They like the
idea of using PlanetMath as a place to demo and test their stuff,
e.g. math search.  They also provide research time to help with
interesting questions.  For example, on the technical research front,
it will be interesting to figure out how to get terms from arXiv, so
we can link to them, not just from them.

\paragraph{Through Deyan, we have links to NIST and the DLMF project.}  We
have the opportunity to collaborate with NIST and create an
integration platform, with PlanetMath being the hub.  We could pitch
this as a sort of "Facebook for math."  NNexus would give us an edge
on Wikipedia on this front.  We can think about getting a NIST senior
on the PlanetMath advisory board.  This may be the most realistic
opportunity for a new partnership.

\paragraph{Support for problem solving may have a meaningful impact on learners.}
If we approach the MAA or others with evidence that what
we're doing is actually helping teachers and students, then we can make
a better impression than if we just blab at them about ``free math
encyclopedias''.  More broadly, remember Pierre's comment that we have to talk
about something exciting in order to bring people in.  For educators, there's nothing more exciting than \emph{results}.  We'd really like to be able to 
say something like ``We have these features, these are what they are
useful for, and, look, it really works!''

\paragraph{There are many possibilities for partnerships.}  Boundless is one
organization that's taken a proprietary software, open content
approach.  They might not want to partner with us, but they are one
``success story'' and we can think more about ``things like this''
where we could have an impact.  Our advanced content is a
``distinguishing difference'' here.

\paragraph{There are many possibilities for innovation.}  Here are a few
that stand out: 
\begin{itemize}
\item[(1)] Math expressions are one content type which is completely
  untouched by competitors and has great potential for interactive
  services.  Competing sites use MathJaX at best, images as a rule of
  thumb, and essentially serve ``dead'' formulas to their readers.  We
  have the expertise to take an edge here.  Add to that diagrams and
  plots, and you get something very exciting.  The June release of
  \LaTeX ML will support LaTeX to SVG graphics conversion, and that
  will allow the creation of a new breed of interactive services.
\item[(2)] We mentioned bibliographies briefly above.  We don't need
  to rediscover the wheel - CiteSeer, Google Scholar, DBLP and many
  others already offer extensive bibliographic references (generally
  with BibTeX support).  We can enable easy authoring with and
  integration of those resources in PlanetMath.  This will also be one
  obvious place to improve our Semantic Web standing, and may be the
  beginning of a detailed ``catalog'' of the world's mathematical
  knowledge.
\item[(3)] There are plenty of active services for e-learning that do
  not exist on the web yet -- creating background reading lists and
  semantic navigation diagrams are two that relate to bibliographies.
  We have already developed some of basics with the new
  ``collections'' functionality.  We should pioneer here.  One obvious
  low-hanging fruit would be to add PDF export from collections.
\end{itemize}

\section*{Threats}

\paragraph{Since we don't have a lot of resources, if someone has enough
  money to get a team together, they can do what we're interested in
  in 1/10th the time.}  For example, this has already happened with
Wolfram|Alpha, which provides a "worse is better" system when compared
with what we hope(d?) to build with the Hyperreal Dictionary of
Mathematics project.

\paragraph{Wikipedia is the 1st thing people will know and the first thing
  people will think of to use.}  Even though they happen to be general
purpose, they are also improving the math side, and are working to
make the whole site easier to use.  We'll really have to emphasize
that we're not \emph{just} an encyclopedia if we want to stay
meaningful.

\paragraph{Other old technologies may be enough for many people.}  Libraries
and papers will probably satisfy most people who currently have access
to them, unless we are much better.

\paragraph{``Stasis'' could happen at any time unless we're fairly careful.}
In particular, just having an open source software platform doesn't
mean a whole lot unless we work to build an active contributor team.
In order to attract new users and developers, we should make our
free/libre software stack as public and attractive as possible.  New
Planetary deployments should be welcomed and supported
(e.g. university-internal course spaces for professors, personal
research setups, publishing production environments, etc.).  The
current effort to refurbish NNexus follows the same line of logic --
make it easy for people to adopt, reuse and contribute.

\paragraph{We are probably still ``the threat'', and we have to be careful.}
Keep in mind the Aaron Krowne and Aaron Swarz lawsuits and criminal
proceedings.  We've learned to watch out for is fatuous lawsuits and
charges.  At the same time, if we're going to be the threat, we should
be an effective one, and build something that works!  Relatively
conservative mathematicians might \emph{use} the system, even if they
wouldn't contribute to development.

\paragraph{The two biggest threats are losing the community and losing the
  small developer team.}  The only viable way forward is to make
PlanetMath an exciting and welcoming place to work, and make Planetary
an exciting and viable FLOSS project.

\paragraph{PlanetMath lost ground over the last two years, so we're fighting
  an uphill battle now.} While we worked to introduce a new platform,
from a user's perspective the site was going down hill, and there was
a pretty major data loss problem.  As a result of these things and
successful competition that was able to ``do one thing and do it
well'', the community shrank.  Google and Alexa rankings reflect this.
We'll need to fight to stay current.  We need excited users and devs
to help us get there.


