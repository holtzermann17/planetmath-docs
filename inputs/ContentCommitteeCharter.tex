
\subsection*{Preamble}
This document serves as the charter for and supplementary operational details of the PlanetMath Content Committee, an official committee of the PlanetMath board.

This charted is not meant to lay out thorough procedures for accomplishing the goals of the committee; instead, part of the job of the committee will be, by experimenting, to understand and codify procedures that work in the context of the PlanetMath community, and formalize them later. This includes things such as what improvements would provide the most benefit as well as what kinds of inducements will work best to standardize the quality of PlanetMath submissions. This document, then, is meant to provide a basic operational framework and overall direction for the Content Committee.

The document may be amended at any time, at the recommendation of the Content Committee. The PlanetMath Board will once again be required to approve any such changes. Unless clearly marked as drafts of proposals or changes, all of the content of this document should be considered official operational policy.

\subsection*{Purpose}
The mission statement of the Content Committee is:

\begin{quote}
\textbf{The Content Committee will preserve and maintain the integrity and quality of the mathematical content and organization of PlanetMath.}
\end{quote}

To this end, the committee will engage in activities designed to improve the accessibility of information, improve the accessibility of the site to new members, improve the content, style, and correctness of individual entries, and ensure that other aspects of the site such as the request list are adequately maintained. In order to do this, the committee will need a substantial active membership, which will be recruited from active PlanetMath members.

\subsection*{Charter}
\subsection*{Organization}

The membership of the Content Committee will be determined by the Board of PlanetMath. When the board determines that additional member(s) is (are) required, the Content Committee will seek to recruit new members.

The Content Committee shall in general seek to find new members who are not already PlanetMath board members.

The Content Committee may set out procedures and policies regarding Content Committee recruitment and tenure. However, all new members and forcible Content Committee ejections must be approved by the Board.

\subsection*{Content Committee Documents}

In order to carry out PM content-related responsibilities/tasks, the PM CC may develop specific policy documents. Below are some examples of ``Content Committee related'' documents:

\begin{enumerate}
\item Standards of PM Content with respect to its Encyclopedia, Paper, Exposition, and Book Sections
\item Enforcement Rules with respect to PM Content (which does not include individual behaviors in the PM Forum)
\item Rules and Regulations regarding PM Request System and PM Orphanage
\item PM Point System
\end{enumerate}

Except on matters having to do with the suspension of a PM user or forcible removal of a Content Committee member, tasks spelled out by these documents may be executed completely within the discretion of the Content Committee.

\subsection*{Primary Content Committee Responsibilities}

Areas of work that are currently seen as important include (but may not be limited to) the following:

\begin{enumerate}
\item Developing/maintaining the standards for PlanetMath content
\item Improving individual PlanetMath entries in its Encyclopedia, Book, Paper, and Exposition)
\item Developing topic areas
\item Developing/improving site and user documentation
\item Managing the PlanetMath Request list and Unproved Theorems list
\item Improving categorization and other meta-attributes of entries.
\item Developing software recommendations for improved content authoring and editorial functions.
\end{enumerate}


\subsection*{Accountability and Documentation}
The Content Committee will provide the Board such information as it
demands to make decisions on Content Committee requests.


\subsection*{Detail}

\subsection*{New Membership Recruiting Procedure}

The procedure for adding members will be as follows:

\begin{enumerate}
\item The content committee will post a note on the PlanetMath forums asking for nominations, or for self-nominations. A nomination shall consist of an application per format specified by the Content Committee in the announcement.
\item A reminder post will be made one week later.
\item After two weeks, existing nominations will be collected and evaluated by the current content committee.
\item Names of successful candidates will be forwarded to the Board for final approval.
\end{enumerate}

\subsection*{Content Committee Responsibilities}

Some of the responsibilities mentioned above are described in more detailed below.

\paragraph{Standards for PlanetMath content}
The \texttt{\htmladdnormallink{PlanetMath Community Guidelines}{http://planetmath.org/?op=getobj&from=collab&id=137}} provides in Section 4 a preliminary discussion of standards for PlanetMath entry content. The committee shall codify these standards into a site document; the content should be specific enough that it is possible in most if not all cases to say easily whether a particular entry conforms to the standards or not.

Another aspect of standards development for PlanetMath is coming to agreement on the discussions that have recently been quite active in the forum regarding the place and manner of copying as a means of creating PlanetMath content.

\paragraph{Improvement of individual entries}
PlanetMath.org trusts in the civil behavior of its members, and we all hope that the members can work together and collaborate as a team in the honorable goal of creating ``math for the people, by the people'', but always under the standards detailed in the \texttt{\htmladdnormallink{PlanetMath Community Guidelines}{http://planetmath.org/?op=getobj&from=collab&id=137}}. The members can (and should) propose corrections to entries via the system in place, and can also discuss changes of entries in the forum. However, if the usual channels of cooperation fail (corrections and forum discussion), an arbitrage by a third party may be necessary. The Content Committee provides this kind of arbitrage (and again, it should always be the last resort) in order to ensure that the content of PlanetMath follows the standards mentioned above. In most cases, the Content Committee will produce a set of recommendations, suggestions and constructive criticism that will be passed on to the author of the entry in dispute. If the author refuses to make the suggested changes or does not comply with the recommendations, then the Committee is authorized to take further action.

In addition, the committee may undertake a review of existing entries in order to ensure general compliance with the standards developed in the previous section. This review would result in a series of corrections and/or other comments that would then be handled by the community in the same way as described above.

\paragraph{Development of topic areas}
In general, the ease of finding information about a particular topic on PlanetMath varies widely. For some areas, there are excellent overview articles; for others, there are not. In the past, there have been efforts to focus on particular areas of mathematics and to build them out in PlanetMath; for example, see the \texttt{\htmladdnormallink{real number project}{http://planetx.cc.vt.edu/AsteroidMeta/Real_numbers_on_PM}}. That project has made little progress because there has not been critical mass to support it. The Content Committee will identify and champion the development of this and other topic areas in order both to improve coverage and to make it easier to find concentrated information about a particular area.

\paragraph{Development/improvement of site and user documentation}
There is little accessible ``new-user'' style documentation regarding the mechanics of, and restrictions on, writing LaTeX that will be acceptable to the system, community guidelines, usage of graphics, link control, and the like. If PlanetMath is to be more widely used and accepted, such documentation is necessary in order to enable those unfamiliar with the software to more easily submit content to the system.

\paragraph{Management of the request list and unproved theorems list}
The title of this section speaks for itself. The request list in particular is quite long, and many of the requests either have been fulfilled but not marked closed, or are poorly defined. The list should be pruned, and requests that have been fulfilled should be marked as such. This will require members of PlanetMath who are knowledgeable in many different areas of mathematics, in order to do a responsible job of assessing site content vis-a-vis the requests.
