%%%%%%%%%%%%%%%%%%%%%%%%%%%%%%%%%%%%%%%%%%%%%%%%%%%%%%%%%%%%%%%%%%%%%%%%%%%%%%%%
\subsection{Introduction}
Now that we have a \emph{flyer} to distribute that describes our work,
I'd like to return to the interesting paragraph from the flyer that
talks about our work plan.  
\begin{quotation}
Our research and development team is working hard
towards implementing new features such as a
centralized bibliographic database, a toolchain for
retrodigitizing public domain mathematical works, a
real-time collaboration environment, and a course
management system.
\end{quotation}
Ray's "3C" list (introduced at our morning session at Columbia, and
soon to be described in more detail here) provides an expanded view,
and my notes on that are in the next section.
A quick check: do the examples mentioned above fit into the proposed
categories?
\begin{itemize}
\item[] Catalog $\rightarrow$ bibliographic database (check)
\item[] Content $\rightarrow$ public domain works (check)
\item[] Community $\rightarrow$ real-time tools (check)
\item[] Software $\rightarrow$ course management system (hm...)
\end{itemize}
So, "Software" seems like a worthwhile 4th category.  Once built, and
in use, the course management system would support aspects of the
"3C's".  Also, "Community" has some natural subcategories: Organization, and Collaborators.
So we have some non-trivial projects here, in 3 or more categories.
But who will do what, and when?
Let's think about the items here and add some "major milestones" that give reasonable projections.  Some likely milestones are:
\begin{itemize}
\item "Fundraising feature complete"
\item "Outreach target 1: research into fundraising possibilities" - check in with the various orgs listed below, find out whether there are reasonable possibilities of working together.  Gauge response to flyers
\item "Books demo" Get a reasonable demo of our OCR to web stuff up (including problem sets)
\end{itemize}

%%%%%%%%%%%%%%%%%%%%%%%%%%%%%%%%%%%%%%%%
\subsection{Focus for this quarter}
\begin{itemize}
\item \emph{Software subproject} [@Joe: By {\bf SEPTEMBER}: Get
the "Switchover" milestone completed, and try
to build up a Planetary user base with
teachers \& departments, see
\href{https://github.com/cdavid/drupal_planetary/issues?milestone=2&page=1&state=open}{the list of issues} for details]
\item \emph{Content subproject} [@Joe, @Ray: by {\bf SEPTEMBER}: Post
 some demo textbooks, problems,
 solutions.  @Joe, @Deyan, @Ray: By {\bf DECEMBER}:
 Ask the Bremen people to provide a hosted
 solution for this to play with (as a
 fallback, by January, implement a Git interface to
 Planetary)]
\item \emph{Community subproject}
 [By {\bf SEPTEMBER}: 
@Ray: \emph{Prepare for meeting with Math
  Museum people, Pierre etc., build some kind of pamphlet},
  cross over with the "news" box, @Joe: write some new user guide
 how-to stuff; by {\bf DECEMBER}
  fundraising drive on PM, aim to raise say \$10K,
  which would allow us to pay small bounties and
  stipends; {\bf JANUARY} talk with NIST about our stuff; {\bf MARCH} build a for-fee
  \href{http://metameso.org/~joe/docs/bce.pdf}{tutoring service}]
\item \emph{Catalog subproject} [@Ray: create list of first books to put through OCR]
[@Ray: obtain LOC card catalog for math and enter into BKN]
\item \emph{Organizational subproject} [@Ray: By {\bf DECEMBER} get the organizational
docs in order.  Note: Joe created a "snapshot" of a lot of this content.
(\href{http://metameso.org/~joe/docs/hhgtpm.pdf}{PDF})
(\href{http://metameso.org/~joe/docs/hhgtpm.tex}{TeX}).]
\end{itemize}

%%%%%%%%%%%%%%%%%%%%%%%%%%%%%%%%%%%%%%%%%%%%%%%%%%%%%%%%%%%%%%%%%%%%%%%%%%%%%%%%
\subsection{The plan}
The subsections below are: Content, Catalog, Community,
Possible Collaborators, and Software.

%%%%%%%%%%%%%%%%%%%%%%%%%%%%%%%%%%%%%%%%
\subsection*{Content}
\begin{itemize}
%%%%%%%%%%%%%%%%%%%%
\item Post textbooks, problems and solutions.  [@Ray, @Joe: (A week or so from
now, once collections functionality is ready.)]
\item Run first few books from queue through OCR. [@Ray: (First batch should be done a week from now.)]
\item Proofread textbooks, starting with half-done calculus book.
[@Ray: (Sometime within a month as spare time shows up.)]
\item Bookshelf, Text, Semantic Tex, Hypertext [@Joe, dev team: By {\bf MARCH}: Get
   an sTeX mode working on PlanetMath]
%%%%%%%%%%%%%%%%%%%%
\item \emph{Commons}
  \begin{itemize}
  \item entries
  \item books
  \item articles
  \item theses
  \item REU reports
  \item problems
  \end{itemize}
%%%%%%%%%%%%%%%%%%%%
\item \emph{Uncommons}
  \begin{itemize}
  \item private workspaces
  \end{itemize}
%%%%%%%%%%%%%%%%%%%%
\end{itemize}

%%%%%%%%%%%%%%%%%%%%%%%%%%%%%%%%%%%%%%%%
\subsection*{Catalog}
\begin{itemize}
\item Locate first set of textbooks to process, begin to define
  "queue" for further processing.
\item books, articles [@Joe, @Ray: By {\bf DECEMBER}: Get Jim Pitman's system
   installed, and get it interfacing to Planetary]
\item web links [@Deyan: By {\bf DECEMBER}: Get NNexus working well, get
   Magdalena's content imported]
\item PM Xi
\end{itemize}

%%%%%%%%%%%%%%%%%%%%%%%%%%%%%%%%%%%%%%%%
\subsection*{Community}
%%%%%%%%%%%%%%%%%%%%%%%
\begin{itemize}
\item user workspaces, accounts [@Joe, @Ray: By {\bf MARCH}: Copy the 750words.com
   Patronage model and use that in place of Google Ads,
   using a drupal module, provide transparency about what
   the money is going for]
\item Social networking programs
[@Ray Set up copy of Friendica to play with on linode.]
\item math blogs
When I (Ray) read the description, it sounded like Friendica had
support for blogs, need to double check that.   Also, whatever software
we use for blogs, will need to have TeX support.
\item chat etc                   
\item subchanneling
\item Prepare flyers, business cards, etc. for potential meetings
with potential supporters.  [@Ray: (Mostly done)]
\item Meet with potential supporters.  [@Ray, @Pierre @Aaron: (Sometime in next two weeks; exact dates not yet forthcoming.)] 
\item Distribute flyers up in local universities, libraries, coffee shops, etc.  (Flyers come in from printer in a week; bulk of distribution in middle of September after new car arrives.)
%%%%%%%%%%%%%%%
\item {\bf Organizational stuff} [@Ray, @Joe: within the next month get through a basic re-org of the organizational stuff]
  \begin{itemize}
  \item Update and rework site documents.  (Steady work in next month or two.)
  \item Organize organizational documents as per post of 8/31/09. (Steady work in next month or two.)
  \item Write 3C and similar planning documents. (Steady work in next month or two.)
  \end{itemize}
%%%%%%%%%%%%%%%
\item \emph{Collaborators (and possible collaborators)}
  \begin{itemize}
  \item Math Museum [@Ray (is this done?): By {\bf DECEMBER}: Prepare for meeting with Math Museum people, Pierre etc., build some kind of pamphlet,
    cross over with the "news" box.]
  \item Project Gutenberg
  Get on gutvol-d mailing list and listen to figure out what they're interested in,
  who's active, etc. and ask about being listed as partner distributing math books.
  \item Distributed Proofreaders
  When we have at least a preliminary proofreading platform and book queue in place,
  contact them to ask if they would be interested in helping with proofreaing.
  \item Creative Commons
  \item IAOL
  Their card catalog is available for free download so we could add it to bibliographic
  database.  Before doing that, however, might be worth seeing if they add anything 
  extra to what we already can get from LOC.
  \item LOC
  Download q section of card catalog,
  \item FTG
  \item Peeragogy
  \item Math for America
  \item \href{https://simonsfoundation.org/}{Simons Foundation}
  They fund MoMath talks, so maybe we can get at them by
  following Pierre's lead.
  \end{itemize}
\end{itemize}

%%%%%%%%%%%%%%%%%%%%%%%%%%%%%%%%%%%%%%%%
\subsection*{Software}
\begin{itemize}
\item Help prepare the new Urbana server with Planetary, SVN, GIT, and BKN.  [@Ray: (Expect delivery within a month.)]
\item Planetary [@Joe: By {\bf SEPTEMBER}: try to build up a Planetary
  user base with teachers \& departments]
  \begin{itemize}
  \item Future development of Planetary: see 
  \href{https://github.com/cdavid/drupal_planetary/issues?milestone=1&page=1&state=open}{the list of issues} for details
  \item Planetary + Git [@Joe, @Deyan, @Ray: By {\bf DECEMBER}: 
    by January, implement a Git interface to
    Planetary)]
  \end{itemize}
\item Bibsonomy / BibKN [@Ray: By {\bf DECEMBER}: Get the bibliography
  software installed on the Desktop SC and populate w/
  MARC files]
\item OJS [@Ray: By {\bf JANUARY}: "Open Journal System" is
  now set up, but now need people to contribute,
  particularly to capitalize on the timely aspects of a
  new preprint server]
\item infty OCR
\item sTeX
\item NNexus
\item Friendica
\item VNC, Desktop sharing
\item Etherpad \& Etherpad LiteR
\item Mumble
\end{itemize}

%%%%%%%%%%%%%%%%%%%%%%%%%%%%%%%%%%%%%%%%
\subsection{Discussion}
For the moment, I can provide a couple of additional related links
that you may like to peruse:
\begin{itemize}
\item \href{http://campus.ftacademy.org/wiki/index.php/Seed_Project:_Planetary}{A "profile" of the Planetary project} -- talking about how we might
find further uses and contributors
\item \href{http://en.wikiversity.org/wiki/User:Arided/Paragogical_Profile}{Joe's 
personal learning profile} - talking about what I'm working on and what might be
coming up for me in the next decade
\item \href{http://peeragogy.org/peeragogy-org-roadmap/}{A roadmap for the ``Peeragogy Project''} - which breaks the upcoming work up into reasonable sections, and provides a memento showing wich goals that have been accomplished
\item \href{http://metameso.org/~joe/thesis-outline.html}{Some figures} 
that show how PlanetMath's activities can be decomposed into categories. 
Do the activities or (or categories) discussed in this document map
into those categories at all?
\end{itemize}
It would be great to make sure that that we have something with this much or more detail for PM.
