\subsection{Preamble}

During the conversation with Joe and Deyan this afternoon, some of the
proposals for NNexus reminded me of the PM-Xi project which Joe and I
had discussed a few years back, so I suggested that looking back at
that proposal might be useful. After the conversation, I looked
through my records and retrieved the document which contained the
final form of this proposal. Originally, I was just gong to sennd out
the PM-Xi portion of that document but, after reading it over, I
decided that maybe the whole document is worth looking over again
now. Basically, this ``dreamlist'' outlines a coherent, comprehensive
vision for a new improved PlanetMath which incorporates the various
projects and proposals which we had been discussing in and around our
summit in Madison. Now that we have rolled out the new platform and
are again looking at the bigger picture, I think it might be worth
pulling his document out, dusting it off, looking at it, and making
use of it.  Whilst some of the details there need to be updated to
reflect our current software platform, by and large I think that the
ideas and plans proposed there are as relevant and useful now as they
were in 2008, maybe more so now that we are in a position to start
implementing them. -- \emph{Raymond Puzio, March 23, 2013}

\subsection{Introduction}

From its inception in 2001. the primary focus of PlanetMath has been on
the production of free mathematical content, chiefly in the form of
encyclopaedia entries. While I think that PlanetMath has been successful
in this area and should continue doing this, at the same time, I think
that it should broaden its horizons to indexing and presenting content
produced elsewhere not only because this would allow it to carry out our
mission more fully but also because, being uniqiuely situated to fill a
role which is only now emerging in the landscape of online mathematics,
it is presented with an opportunity which, if seized in a timely
fashion, could launch the organization into a position of leadership
within this landscape.

When PlanetMath started seven years ago, it found itself in a unique
role. At that time, the internet was just becoming available to the
general public and most content was in the form of individual web pages
or suites of pages written by at most a handful of users. At that time,
the notion of a website writtten collaboratively by a sizable community
of authors and the software platform underlying such an effort were
quite novel ideas and there were at most a handful of other sites which
oferred large quantities of mathematical content. As time went on and
the novelty wore off, others have done similar enough things to the
point where we no longer stand out. Collaboratively written sites are
now quite common, web encyclopaedias are almost as common as people
posting pictures of their pets, and there is plenty of mathematical
content available from a number of sources. To be sure, we still have
the distiction of having been first along with the concommitant
experience and advantages of our headstart, but our position now is, at
best, first among equals, not as a forerunner far in front of the crowd.

Exactly this proliferation which no longer makes us as distinctive as we
used to be has created a new opportunity for us to lead rather than lag.
The mound of mathematical knowledge now available online is crying out
for organization. Presently, the tool used for navigating this sea of
assertions is the general purpose search engine, which is limited by its
inability to understand specifically mathematical search criteria.
PlanetMath could improve on this effort tremendously by channeling its
collective expertise into CBPP projects which provide the data needed
for intelligent searching of mathematical content and projects which
directly organize and catalog content found in various sources.
Undertaking such a project is not entirely a new direction because
already the books section presents content which is primarily written
elsewhere; rather, it is a matter of doing this on a much larger scale.

The scenario described here is analogous to the evolution of the Great
American Shopping Experience. In the early days, one had
mom-and-pop-shops. Typically, these were located in a downtown business
district or along a main street and each would specialize in some types
of merchnadise such as stationry, hats, or candy. Then there came the
department stores and supermarkets (not to mention combinations of both
in a single business), which combined these various specialties under
one roof in a unified fashion and a single check-out. Then there came
the shopping mall which combined features of both earlier types. Built
around one or more superstores, it also had places for smaller specialty
stores, evoking the appearance of the row of storefronts along a street.
By analogy, individual websites are like mom-and-pop-stores, offering
some specialty such as Fibonnacci numbers, the reaearch interests of an
individual, or a math department of a university. PlanetMath is like a
department store offerring books, entries, and discussion on various
areas of mathematics in one place. The next level which I propose here
would be like a mall insofar as it would collect not just offerrings
provided by the PlanetMath community, but also material from elsewhere
in one location. Just as shopping malls offer the convenience of buying
all sorts of things in one place rather than having to travel from place
to place, so too this expanded version of PlanetMath would offer the
convenience of being able to take care of most of their mathematical
needs from within a single website. Just as erecting a shopping mall
requires entails providing a building with stalls for shops but not
furnishing the merchandise, so too the proposals described below calls
for providing the infrastructure for hosting, indexing, and displaying
various types of content, but not for generating the content, which
instead will be harvested, transcluded, and linked from various other
sources.

By and large, the platform described here is based upon already
well-known and established software architectures. In many cases, it
should simply be possible to simply deploy already written programs, and
where this is not possible, it should usually be a matter of writing a
program along the lines of what has been done many times before as
opposed to developing something new. Using familiar formats such as
drag-and-drop desktops, social networking, etc. should make it easy for
most new users to get started. Rather, most of the novelty here comes in
integrating these genres and adapting them to the needs of
mathematicians. For instance, one could go from a social networking
directory to a math blog with entries in TeX describing progress on a
research project to a desktop where one can participate in furthering
that research. Having said this, it is worth noting that there are
definitely ways in which new technology could be incorporated in a
useful way. Most notably, the scholium system should provide a versatile
foundation for implementing (or re-implementing) many features in a more
flexible way. Likewise, relativistic databases, proof checkers, parsers,
and the like could all be incorporated into this scheme in appropriate
places to enhance users' capabilities.

\subsection{Mathematical Web Directory}

Goal: To present links to as many webpages about mathematical topics as
possible, ideally all of them, along with metadata and descriptions of
the same. Examples of types of pages which should go in such a directory
are as follows: web sites and pages about mathematics, mathematical
organizations' web pages, homepages of mathematicians with publications
and similar mathematical content (as opposed to homepages of
mathematicians which say nothing or next to nothing about mathematics),
homepages of mathematics departments and institutes, open access math
journals and preprint servers (and maybe also closed ones).

\begin{itemize}
\item
  Provide links to web sites carrying mathematical content
\item
  Provide permanent links when possible.
\item
  Links and data will be added by users.
\item
  Each entry will have a description..
\item
  There will be various metadata fields for ease of reference.
\item
  Work towards time when, not only will most mathematical web sites be
  listed here, but people will routinely look in this section to find
  websites, and hence people with websites will be interested in making
  sure their sites are listed.
\item
  Write letters to webmasters whose sites are listed. ** Promote
  goodwill and accuracy by having them approve description. ** Encourage
  linking to PM.
\item
  Provide unified interface to various web sites and services.
\item
  Work metadata into a metric-based search engine.
\item
  Provide space for user reviews and comments.
\end{itemize}
\subsection{PlanetMath Cross-index (PM-Xi)}

\paragraph{Goal} To mutually reference as much of the mathematical literature as
possible in a structure which cuts across mathematical works
transversally to link portions which discuss the same subject material
as well as to provide local metadata which helps index the various
topics discussed in the literature for ease of retrieval. This would
provide a directory to the literature in the form of a certain type of
basic linking between works in a standardized format which later could
be augmented with descriptions of the links and other types of links as
well as be of use in compiling more specialized directories such as
annotated bibliographies as well as generally in accessing and using the
literature.

\paragraph{Ontology}

\begin{itemize}
\item
  The cross-index consists of factoids, loci, and allusions.
\item
  A factoid is an irreducible unit of mathematical exposition, e.g. **
  Statement of a definition. ** Statement of a theorem. ** Proof of a
  theorem. ** Example
\item
  A locus is a portion of a mathematical text which expresses a factoid.
\item
  An allusion is a portion of a mathematical text which refers to a
  factoid but does not state it because it is assumed to be known by the
  reader or the reader is referred elsewhere for an explanation.=
\end{itemize}

\paragraph{Tables}

\begin{itemize}
\item
  Factoid table ** Metadata --- as a first approximation, we can take
  the metadata fields attached to PM entries. ** Loci --- A list of loci
  in which the factoid in question is described.
\item
  Locus table **Genre  This column describes the type of work in
  which the item occurs, e.g.~book, journal article, web page, thesis,
  lecture notes= , conference proceeding, etc. ** Work  The
  particular work in which the item occurs. ** Location  Am
  indication of exactly where in the work to find the item. The exact
  form of data in this slot will depend upon the genre. As far as
  feasible, this should be in a form which enables a computer to select
  and highlight the locus in question from a digital version of the
  work.
\item
  Allusion table **Genre  This column describes the type of work in
  which the item occurs, e.g.~book, journal article, web page, thesis,
  lecture notes= , conference proceeding, etc. ** Work  The
  particular work in which the item occurs. ** Factoid --- The factoid
  to which the allusion is made. ** Location Am indication of
  exactly where in the work to find the item. The exact form of data in
  this slot will depend upon the genre. As far as feasible, this should
  be in a form which enables a computer to select and highlight the
  locus in question from a digital version of the work.
\end{itemize}

\paragraph{Workflow}

\begin{itemize}
\item
  Where possible, work indexing into other activities. ** Have loci
  automatically be generated from TeX code, e.g.~make a theorem locus
  automatically whenever a theorem environment is encountered. ** Set up
  text viewer so that readers can add indexing on the fly. ** Allusions
  are suggested on the fly by the autolinker; subsequently, readers will
  confirm whether they were chosen correctly and automatically add
  allusions to the database as appropriate.
\item
  As they are found by users, loci and allusions could go into a pool
  for later attachment to factoids.
\item
  Users would own and maintain factoids much as they already maintain
  entries.
\item
  There would be an inbox for suggesting loci to be added to a factoid
  attached to that factoid.
\item
  One could file a correction to a factoid to point out mistakes or
  omissions in metadata or inappropriate loci.
\item
  The maintainer of a factoid will be expected to answer corrections and
  proopsals for extra loci in a timely fashion..
\item
  There will be a place for associating unattached allusions with
  unattached loci; upon attaching the loci to factoids, the allusions
  associated to them would automatically be attached to the factoids in
  question,.
\item
  Since it is possible that two users may inadvertantly create two
  copies of the same factoid with diffferent loci, there will be a
  mechanism for merging factoids.
\end{itemize}

\subsection{Improved Papers and Exposition Sections}

\paragraph{Goal} In addition to improving facilities for users to add their own
publications, inculde pointers to papers found in online journals,
preprint servers, private and institutional web pages, etc. Add value to
the collection by aggregating various content, adding user comments and
linking entries.

\paragraph{Types of mirrored, transcluded or linked content}

\begin{itemize}
\item
  Preprints
\item
  Open access journals
\item
  Theses
\item
  REU Reports
\item
  Online seminars from IRC channels and elsewhere
\item
  Recorded lectures
\end{itemize}

\paragraph{Added value to content hosted elsewhere}

\begin{itemize}
\item
  Classification and other metadata
\item
  Metric-based search customized for mathematics.
\item
  Comments, discussion, introductions, and explanations by users
\item
  Annotated bibliographies and directories
\item
  User-supplied reviews and surveys
\item
  Ratings and recommendations by users
\item
  Automatic linking
\item
  Bibliographic linking, with clickable links to pull up linked articles
  when available.
\item
  Standardization across different platforms of infromation providers.
\item
  One stop shopping
\end{itemize}
\subsection{Improved Books Section}

\paragraph{Goal} To go beyond simply listing books to making a virtual reading room
which serves as a locus for scholarly activities centred about math
books.

\begin{itemize}
\item
  Have users add value via discussions, introductions, notes, etc.
\item
  Build a community discussing books around the collection.
\item
  Have platform for creating, maintaining, reading, and searching the
  following sorts of subsidiary works: ** Reviews ** Introductions **
  Literature guides / Bibliographies
\end{itemize}

\subsection{Improved Collaboration Section}

\paragraph{Goal} To make it easy for members to work together on projects of common
interest, both in researching them and in writing up the results of
research. Collaboration objects will serve as a central focus in which
various resources relevant to a particular project can be centrally
located on a shared desktop. The available resources come in several
different types, it being up to the collaborators to choose ones which
they find useful for their purposes. There are communications resources,
such as discussed in the section on networking, which could be useful
for discussing work. Computer algebra systems, proof assistants,
numerical packages, and the like could be useful for carrying out
computations. Editors, linked information managers typesetting programs
could be useful for recording and presenting results.

\paragraph{Design considerations}

\begin{itemize}
\item
  Lifecycle of a mathematical collaboration can roughly be divided into
  three phases: ** Brainstorming and collecting information **
  Calculating and proving results ** Writing up results as journal
  article, conference proceedings, book, etc.
\item
  Different software utilities help with these phases: ** For
  brainstorming, communications platforms, mind mappers, and note
  organizers. ** For calculation, numerical packages, graphic packages,
  algebra systems, and proof asssistants. ** For writing up results,
  editors, typesetters, and slide makers.
\item
  Smooth transition between these stages. ** Utilities to translate
  calculating formats to typesetting formats. ** Ability to annotate
  files with links and notes showing flow of ideas and work.
\end{itemize}

\paragraph{Communications channels}

\begin{itemize}
\item
  These will be synchronous and asynchronous. They should be integrated
  and the synchronous modes should have logging capability in order to
  be treated uniformly with the asynchronous modes after the fact. **
  Asynchronous modes: \emph{*} E-mail \emph{*} Forum \emph{*} Weblog **
  Synchronous modes \emph{*} Whiteboard \emph{*} Text chat \emph{*}
  Audio \emph{*} Video (still and moving)
\item
  Communications channels logs should be organizable by time and have
  room for adding commentary.
\end{itemize}

\paragraph{Storage}

\begin{itemize}
\item
  Each collaboration will come equipped with its own storage organized
  as a hierarchical file system.
\item
  Each collaboration will have its own name space for its files.
\item
  Files from outside collaboration name spce can not only be accessed by
  appending namespace but can be given local aliases.
\item
  Versioning (after the fashion of CVS) ** Each file can be stored in
  multiple versions. ** Bare name of file will refer to default version
  (typically latest version) but other versions can be accessed by
  appending version specifier to file name. ** Preserves history of
  editing and allows recovering data from older versions. ** File
  history can be forked. ** Utilities to navigate history, editing
  history trees, searching and comparing past versions.
\end{itemize}

\paragraph{Desktop}

\begin{itemize}
\item
  Each collaboration comes with a virtual desktop where collaborators
  can work. This workspace is configurable and persists between
  sessions.
\item
  This desktop is organized as one or more screen layers, each of which
  can hold icons and windows and has a menu sidebar.
\item
  Menu sidebar has program launcher, bookmarks for important files,
  links to collaboration fora and similar communication hubs, and help
  centre.
\item
  Collaborators can drag and drop icons for files and appplications on
  the desktop.
\item
  Various applications are available from the sidebar menu, organized in
  a launcher by type: ** Communications --- e-mail, messages,
  conferencing, etc. ** Calculation --- numerical packages, computer
  algebra, proof assistants, etc. ** Programming --- compliers,
  interpreters, development environments, etc. ** Graphics --- plotting,
  diagramming, graphical editing, curve fitting, etc. ** Text ---
  editors, typesetters, word processors etc. ** Presentation --- slide
  makers, etc. ** Utilities --- file format conversion, scheduling, etc.
\item
  Collaborators choose which applications to put on their launcher from
  a master menu.
\item
  Button on the menu bar for adding or removing applications from
  launcher.
\item
  For convenience, can set up a desktop according to one of several
  ready-made packages, e.g.~numerical, statistical, algebra, etc., each
  of which will have a selection of applications and the like which are
  likely to be used for a certain type of collaborative project.
\end{itemize}

\subsection{Web Services}

\paragraph{Goal} To provide online access through web frontends to working copies
free computer programs of use to mathematicians along with documentation
and auxiliary files.

\paragraph{Types of Services}

\begin{itemize}
\item
  Computer algebra systems
\item
  Autolinker for user-supplied documents
\item
  Bibliographic tool
\item
  Proof assistants / Logic checkers
\item
  Statistical packages
\item
  Numerical packages
\item
  Mathematical graphics
\item
  Scientific, engineering, and statistical calculators
\end{itemize}

\paragraph{Documentation}

\begin{itemize}
\item
  Standardized location for documentation which comes with packages.
\item
  FAQ and README
\item
  Place for user-supplied documentation.
\item
  Place for user feedback and comments.
\item
  Links to package homepages, wikis, sources, etc.
\end{itemize}

\paragraph{Integration}

\begin{itemize}
\item
  Links from entries to packages which implements algorithms described
  there.
\item
  Include computational applets in entries.
\item
  Utilities for converting between formats.
\end{itemize}

\paragraph{Access}

\begin{itemize}
\item
  Each service will be accessible through a webform on its homepage.
\item
  Where feasible, services might be embedded into entries and the like.
  E.g. an entry on ideal basis might have a link to an algebra system
  too compute bases for polynomials of number fields.
\item
  Services can be accessed via icons on desktops.
\end{itemize}

\subsection{Social and Professional Networking}

\paragraph{Goal} To serve and extend the mathematical community by making a virtual
setting for mathematical social interaction. this interaction could take
the form of teaching, lecturing on results, discussing mathematical
topics, etc. This will occur in a virtual space which not only provides
a simulacrum of familiar haunts of mathematicians such as classrooms,
conference halls, libraries, and offices but also extends the
possibilities, in particular transcending spatial limitations (hyperreal
inbstitute of mathematics). However, the relation here is meant to be
complementary, not competitive --- in particular, notable via geographic
data, the system is meant to link to the physical world and supplement
it with extra capabilities as opposed to supplanting it.

In order to bring together like-minded individuals who can make things
happen in this setting, there will be social networking facilities along
the lines of Facebook, MySpace, and similar Web 2.0 mainstays, but
adapted towards mathematicians and integrated with the discussion media
mentioned above. These facilities will allow participants to post
profiles on their homepages, communicate with each other on various
levels from the casual to the familiar, locate compatible individuals
for projects, and set up meetings and collaborations.

\paragraph{Conference facility}

\begin{itemize}
\item
  Multiple coordinated synchronous communication channels ** Chat **
  Simultanously editable text ** Whiteboard ** Internet telephony **
  Video (still and moving)
\item
  Set up channels so that it easy to identify source. ** Have virtual
  meeting table with placeholders of participants around it. ** Use
  different colors or fonts to identify locutors in text-based channels
  (e.g.~chat, simultaneous editing, perhaps different colored whiteboard
  markers). ** Make telephone voices appear to come from different
  locations corresponding to locations around virtual table. ** Place
  video feeds at location of table of their originator. ** In sddition
  to table, have priveleged locations in the form of a virtual podium
  and projector which may be occupied by a particular participant or
  rotate between participants, depending on format of meeting.
\item
  Log these channels in a coordinated fashion. ** Look into voice
  recognition for offline transcription preparation. ** Be able to
  replay recordings online. ** Be able to attach annotations to the logs
  in appropriate places. ** Bundle transcriptions with summarries and
  annotations in some structured format.
\item
  Scheduling ** Calendar for planning ** Automatic invitations,
  reminders, and RSVP for participants.
\end{itemize}

\paragraph{Member directories}

\begin{itemize}
\item
  Locate fellow mathematicians according to various organizational
  schemes. ** Geographical \emph{*} Map of members foldable to different
  resolutions. \emph{*} Locator to find members within given areas. **
  Subject areas \emph{*} List members by subject areas of interest.
  \emph{*} Indicate level of knowledge in various areas. ** Potential
  relationships \emph{*} Student (looking for teacher) \emph{*}
  Teacher.tutor (looking for student) \emph{*} Researcher (looking for
  collaborator) ** Participation in the directory is completely
  voluntary and members can opt in our out out of any portions as they
  desire.
\item
  Matchmaking serivce to suggest relationships based on above.
\item
  Facility for adding comments, ratings, and endorsements.
\item
  Place in directory listing for people to supply biography,
  publications, credentials, and like material to enhance their
  desirability and establish their reputations.
\item
  Have neutral place for people to meet and post messages to each other
  in order to negotiate arrangements.
\end{itemize}

\paragraph{Homepages}

\begin{itemize}
\item
  Central place for a member to post personal profile and face to the
  world.
\item
  Standardized formats and locations for various sorts of information:
  ** Personal profile ** Biography ** Publications and self-archiving **
  Mathematical interests ** Social and professional networking interests
  (i.e.. student, teacher, etc.as above). ** Pictures, avatars,
  signatures, etc. ** Directory information (see above) ** Contact
  information ** Personal tidbits and trivia
\item
  Place to store personal files, such as notebooks from sessions with
  computer algebra system.
\item
  Personalizable desktop for applications used by member.
\item
  Varying levels of privacy:
\item
  Public profile and information for all comers.
\item
  Part of profile only available to buddies to avoid potential abuse by
  anonymous strangers (e.g.~spamming to e-mail address).
\item
  Certain information only available to students, collaborrators, or
  other such groups and individuals.
\item
  Confidential information for one's eyes only.
\item
  This can be implemented via access control lists on everything.
\item
  Incorporate recommendations for homepages from International Congress
  of Mathematicians.
\item
  People can leave public and private messages on homepage.
\item
  TeX will be enabled for messages and other postings to facilitate
  communicating math.
\item
  Members can make various types personal websites devoted to their
  interests accessible from their homepages. ** Individual webpages or
  suites of pages ** Weblogs ** Standardized websites (e.g.~wiki,
  noosphere)\\ ** Directories, annotated biblographies, and similar
  references. ** Discussion fora
\end{itemize}

\paragraph{Groups}

\begin{itemize}
\item
  Members can form associations at various levels from, like, loose,
  informal groups of buddies of a member to more tight and formal groups
  like a virtual calculus class.
\item
  Each group can share resources and a virtual space. ** Group homepage
  and attached websites. ** Desktop (cf. section on collaborations) **
  Mailing list and like media internal to group.
\item
  Some reasons for forming groups ** Virtual classroom ** Collaboration
  ** Virtual seminar or conference ** Study group on a certain work or
  subject area ** Discussion or brainstorming group ** Counterpart to a
  physical association
\item
  Ownership ** Initially owner is founder of group, but can be
  transferred. ** Owner manages membership in group. (This privelege can
  be shared with trusted members.) ** Owner allocates access to
  resources. (This privelege can be shared with trusted members.) **
  Members have access to group resources which may not be availabe or
  only partially available (e.g.~read only) to general public.
\end{itemize}
\subsection{Planetary Mathematical Editions}

\paragraph{Goal} To provide online editions of all mathematical works whose
copyrights have expired as well as more recent out-of-print works whose
authors have granted permission and born-free math books, especially
works derived from content produced by CBPP.

\paragraph{Subgoals}

\begin{itemize}
\item
  Have preliminary editions of all math books in the public domain ready
  by 2012.
\item
  Eventually typeset all these books in a dialect of TeX which encodes
  meaning of equations and layout, not just apperarance on the page so
  as to facilitate uptake into PM-Xi and HDM.
\item
  Offer the finished product in three forms: ** Online version readable
  in situ. ** Electronic version in various file formats. ** Paper
  version via on demand print shop.
\end{itemize}

\paragraph{Fringe Benefits}

\begin{itemize}
\item
  Publicity for PM.
\item
  Viral advertising via books.
\item
  Attracting new members.
\item
  Revenue from sponsorships, dedications, and purchases.
\end{itemize}

\paragraph{Defining differences}

\begin{itemize}
\item
  Focus on math books.
\item
  Interest in unrenewed copyrights.
\item
  Preparing digital editions, not just scanning.
\end{itemize}

\paragraph{Types of works by date}

\begin{itemize}
\item
  Before 1923 --- suffices to check publication date
\item
  1923-1963 --- need to check renewal record
\item
  Afer 1963 --- need to obtain permission of copyright holder
\end{itemize}

\paragraph{Scale}

\begin{itemize}
\item
  Approximntely 5000 math books (ca. 1.5E6 pp.) published before 1923.
\item
  Approximately 10000 math books (ca.3E6 pp.) whose copyrights were not
  renewed.
\item
  To finish the project in a reasonable time, we should have at least
  1000 pp entered daily when project is up to speed.
\item
  Assumming 10 pp.~daily for the average participant, this would require
  100 or so volunteers.
\end{itemize}

\paragraph{Workflow}

\begin{enumerate}
\item Locate a book in the public domain. 
\item Have the book scanned in.. 
\item Run scanned immages through OCR program. 
\item Check output and correct as necessarry. 
\item Release preliminary edition. 
\item Recast into a semantically informed dialect of TeX. 
\item Proofread. 
\item Release Planetary Edition.
\end{enumerate}

\paragraph{Sponsoring preparation of books}

\begin{itemize}
\item
  Price of sponsorship will cover cost of scanning (around 30 dollars)
  and costs of running project (20 dollars).
\item
  Total price of sponsorship is comparable to price of book.
\item
  The sponsor will be named on the title page of the Planetary edition.
\item
  Sponsors will be thanked on the project's donor page by level.
\item
  There will be a page for sponsoring books.
\item
  The unit of sponsorship is a single book.
\item
  Sponsoring a bookshelf means underwriting 20 books.
\item
  Sponsoring a bookcase means underwriting 200 books.
\item
  For sponsoring five books, the donor recieves a paperback version of a
  book (\$15 value for \$250 donation).
\item
  For sponsoring a bookshelf or a bookcase, the donor recieves a
  hardbound version of a book (\$50 value for \$1000 or \$10000
  donation).
\item
  Bookshelf sponsors will be invited, free of additional charge, to the
  annual awards dinner where they will be thanked publicly for their
  generosity with some token of appreciation (e.g.~plaque, certificate).
\item
  Donors can sponsor an edition in memory of someone.
\end{itemize}

\paragraph{Sponsoring project}

\begin{itemize}
\item
  Fund platform, salaries, and other infrastructure.
\item
  Be thanked on donor page.
\item
  Offer differrent packages and incentives: ** Logo and link on all
  pages ** Matching donations --- pay certain amount for every private
  dedication. ** Thank corporate sponsors on title page. ** Sponsorhip
  of new sections, features, and events.
\end{itemize}

\paragraph{Grants}

\begin{itemize}
\item
  Do not depend on them or put too much effort towards them.
\item
  Should an agency or program which fits our objectives well arise, it
  may be worth applying.
\item
  Cultivate relationships with people who grant providers respect.
\end{itemize}

\paragraph{Partners}

\begin{itemize}
\item
  A scanning project (e.g.~Internet Archive)
\item
  An on-demand printer (e.g.~Lulu)
\item
  PG, DP?
\item
  Math digital library projects.
\item
  Free content advocates
\end{itemize}

\paragraph{Workforce}

\begin{itemize}
\item
  The work of editing and proofreading TeX will be done by PM members.
\item
  The editor-in-chief will take care of assmembling the contributions
  into a whole, taking care of whatever technical details need
  attention, and overseeiing the operations.
\item
  Participants will be awarded PM points for their contributions.
\item
  To prevent abuse, points will only be awarded for pages successfully
  edited.
\item
  As an added incentive, someone who has edited 1000 pages will recieve
  a PM mug. (\$10 value, breaking down to a cent per page)
\item
  There will be a prize for the person who has edited the most pages in
  a year.
\end{itemize}

\paragraph{Platform}

\begin{itemize}
\item
  There will be a list of books describing how many pages have been
  checked, how many have been double checked and how many remain to be
  processed..
\item
  By clicking on a book on that page, one arrives at a page listing the
  pages of the book.
\item
  By clicking on a page which is not yet edited in this listing, one
  arrives at a web page which shows scanned input, the rendering of TeX
  produced by Infty, an edit box containing this TeX source,
  instructions. and a button labelled ``Submit''. One can then edit the
  source to make the rendered output match the image, then press a
  button to enter this edit
\item
  Once the page has been entered, its status is changed to ``checked''
  and this new status is then reflected on web pages.
\item
  There will be a page available to the editor-in-chief or people
  trusted by the editor-in-chief for double-checking work. On this page,
  there will appear the original scanned image, a rendering of the
  edited source code, and buttons labelled ``Accept'' and ``Reject''.
\item
  If the editor-in-chief presses ``Accept'', the status becomes
  ``double-checked'', which is reflected appropriately on web pages and
  the proofreader recieves credit.
\item
  If the editor-in-chief presses ``Reject'', the changes are discarded.
  The status reverts to ``unedited'' and the proofreader recieves no
  credit.
\item
  In addition to the listing of pages of books, there will be a button
  somewhere which one can click to have the system assign a page for
  proofreading.
\end{itemize}

\paragraph{Webpages}

\begin{itemize}
\item
  Homepage (static)
\item
  About Planetary Mathematical Editions (static)
\item
  Our Benefactors (dynamic)
\item
  Our Partners (static)
\item
  Bookstore (form)
\item
  Catalogue (dynamic form)
\item
  Reading Room (dynamic)
\item
  Sponsor a Book (form)
\item
  Volunteer (static)
\item
  F.A.Q. (static)
\item
  Feedback (form) ** A box for entering comments ** Contact information
  ** A box for pointing out errata.
\end{itemize}

\paragraph{Architecture}

\begin{itemize}
\item
  Storage: A database for holding information about editing and a
  directory for holding files of pages.
\item
  Backend: A set of commands to manipulate these data.
\item
  Gatekeeper: Only allows authorized users to carryout commands.
\item
  Frontend: Coommand-line, web, mail, graphic and other interfaces to
  the commands of the backend.
\end{itemize}

\paragraph{Tables}

\begin{itemize}
\item
  Books ** Identifier --- Numerical identifier of book within database.
  ** Title ** Author ** LC call number ** MSC classification ** Date of
  original publication ** Date entered into system ** Dedication text **
  Acknowledgement of sponsor of book
\item
  Pages ** Book --- Identifier of book to which page belongs. ** Number
  --- Number of page ** Graphic --- Graphic image of page (or pointer to
  file). ** OCR --- Output of OCR program. ** Edited --- Human-edited
  version of the above, otherwise blank. ** Status --- Either Available,
  Being-checked, Checked, or Double-check= ed. ** Proofreader ---
  Identifier of user who edited the page, otherwise blank.
\item
  Events ** Book added ** Book removed (presumably done editing) ** Page
  assigned for editing ** Page unassigned (e.g.~took too long) ** Page
  edited ** Page edit accepted ** Page edit rejected ** Dedication added
  ** feedback recieved
\end{itemize}

\paragraph{Scalar data}

\begin{itemize}
\item
  Next available numerical identifier for a book
\end{itemize}

\paragraph{Backend functions}

\begin{itemize}
\item
  {\bf list-books}  Generate a list of all books in the system.  Input:
  none.  Output: A list, each of whose items is a list of seven items
  as follows:
  \begin{itemize}
  \item numerical identifier of book 
  \item Title of
    book 
  \item Author of book 
  \item Total number of pages 
  \item
    Number of pages which have been proofread 
  \item Number of pages
    which have been double checked 
  \item Number of pages available for
    proofreading.
  \end{itemize}
\item
  {\bf list-pages}  List all the pages of a book along with their status. 
  Input: Identifier of book.  Output: List of pairs of page numbers and
  their respective status.
\item
  {\bf describe-book}  Return information which the system stores about a
  book.  Input: Numerical identifier of book.  Output: A list of data
  which the system stores about that book.
\item
  {\bf view-page}  View a given page of a given book.  Input: Numerical
  identifier of book, internal number of page.  Output: Graphic file of
  scanned page, current TeX file, rendering thereof.
\item
  {\bf check-out}  Check out a page for editing. The status of the page is
  marked as B and the proofreader has three days to tweak the TeX and
  check it back in.\\  Input: Numerical identifier of book, internal
  number of page.  Output: TRUE (success) or FALSE (failure)
\item
  {\bf check-in}  Check an edited page back in. If the TeX compiles
  properly, the status gets upped to C, otherwised an error message is
  given so one has to try again.  Input: Numerical identifier of book,
  internal number of page, edited TeX file.  Output: A list whose first
  element is one of four keywords indicating outcome:
  \begin{itemize}
  \item SUCCESS
    -- if page was checked out by same user and TeX compiles. Tail of list
    empty. 
  \item NOT-CHECKED-OUT --- if page was never checked out Tail
    of list contains status of page. 
  \item WRONG-USER --- if someone
    tries to check in a page which someone else is working on. Tail of
    list empty. 
  \item TeX-ERROR --- if the TeX does not compile. Tail
    of list contains error messge from TeX program.
  \end{itemize}
\item 
  {\bf abort-edit}  Make a page that was checked out avilable for editing
  again without checking in any edits.  Input: Numerical identifier of
  book, internal number of page.  Output: A list whose first element is
  one of three keywords indicating outcome:
\begin{itemize}
 \item ABORTED -- if page
  was checked out by same user. Tail of list empty. 
\item
  NOT-CHECKED-OUT --- if page was never checked out Tail of list
  contains status of page. 
\item WRONG-USER --- if someone tries to
  check in a page which someone else is working on. Tail of list empty.
\end{itemize}
\item
  {\bf double-check}  Indicate that a checked page has been reviewed by an
  editor so its status can be promoted to D.  Input: Numerical
  identifier of book, internal number of page.  Output: A list whose
  first element is one of three keywords indicating outcome:
  \begin{itemize}
  \item
    DOUBLE-CHECKED --- if page had correct status and user is authorized.
    Tail of list empty. 
  \item INSUFFICIENT-PERMISSION --- if user is
    nor authorized by editor. Tail of list empty. 
  \item NOT-PROOFREAD
    --- if page was not proofread Tail of list contains status of page.
  \end{itemize}
\item
  {\bf reject-edit}  Indicate that an edit is not up to the demanding
  standards of an editor, so its status is demoted to A.  Input:
  Numerical identifier of book, internal number of page.  Output: A
  list whose first element is one of three keywords indicating outcome:
  \begin{itemize}
  \item REJECTED --- if page had correct status and user is
    authorized. Tail of list empty. 
  \item INSUFFICIENT-PERMISSION ---
    if user is nor authorized by editor. Tail of list empty. 
  \item
    NOT-PROOFREAD --- if page was not proofread Tail of list contains
    status of page.
  \end{itemize}
\item
  {\bf assign-page}  Pick an available page for editing. At least in later
  versions, this will be done strategically, so as to favor books which
  have been in the system longer or books in which only a few pages
  remain for proofreading.  Input: None.  Output: Numerical
  identifier of a book and internal identifier of a page in that book.
\item
  {\bf add-book}  Adds a new book to the system for proofreading.  Input:
  A list of data about the book to be stored in the system.  Output:
\item
  {\bf remove-book}  Removes a book from the system, returning the files
  which have been edited. This is to be used when everyone is done
  proofreading and double-checking a work so it is time to release an
  edition.  Input: identifier of book to be removed.  Output: files
  related to book packaged in a convenient format
\end{itemize}

\paragraph{News}

\begin{itemize}
\item
  Newsletter
\item
  Annual Report
\item
  Press Releases
\item
  Announcements of special events
\end{itemize}

\paragraph{Forms}

\begin{itemize}
\item
  Thank-you letter for donation
\item
  Letter thanking volunteers for doing 1000 pages.
\end{itemize}
