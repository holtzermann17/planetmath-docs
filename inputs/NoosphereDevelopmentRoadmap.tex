\begin{itemize}
\item Tweak and test the NewFeatures (including/based on Tags, Roles/Permissions, Queues) that are already implemented.
\item Generalize data export and storage, including an XML-based content backend (see NewDesign)
\item Streamline layout templates and stylesheets to create a new frontend\footnote{\url{http://en.wikipedia.org/wiki/XSL_Transformations}} (using some work done by Springer).
\item Port content to CouchDB.
\item Develop a unified bibliography management system for Noosphere.
\item Document the API, with a command-line interface as a testbed.
\end{itemize}

\subsection*{On the use of Extensible Stylesheet Language}
The object handling code also returns the objects as Perl
objects or HTML strings. This object handling code will be
changed so that all functions return XML representations
of objects. This will allow for XSLT stylesheets to be
used to convert the output from the backend into any
format that the user prefers.

\subsection*{On CouchDB}
CouchDB will likely be the new database backend for
Noosphere because of its document centric nature and
built-in versioning support. As a small test case the new
bibliography managing component of Noosphere will be
written using a CouchDB database and likely completely in
Python (Perl is still the core language for the upcoming
HTTP/XML Interface for Noosphere).

But... At first it seems that CouchDB is slow on insert
and update. I am waiting to see how scalable the system is
for very large object collections. I'm also interested in
how long it takes to create an index for views.
