
\subsection*{Completed}

\begin{itemize}
\item Site membership - It exists! However, the whole membership system isn't that easy to understand, so there may be more feature requests in this area soon. Documentation would help. 

\item Search the forums - It's possible to do this now! However we may have other "targeted search" issues to solve in the future. (E.g. searching within documentation or other non-encyclopedia sections, searching by MSC.) 

\item Expand-all option when viewing messages - This is at least completed as a hack: Add "\&msgexpand=-1" to your PM url when browsing in the forum. (It would be nice to better expose this feature.) 

\item email posting to forums - Completed but not well exposed; we should investigate. 
\end{itemize}

\subsection*{We're working on it}

\begin{itemize}
\item noosphere documentation - See the original feature request: \url{http://wiki.planetmath.org/AsteroidMeta/Noosphere_documentation}. (This wiki is going to contain up to date documentation of the software and development practices.) 

\item Planetmath documentation - We're also working on documenting other parts of the site/nonprofit better. 

\item "create a section of PlanetMath? where everybody can post just any content" 

\item Accessible/text-only webpages - We should double check this, but perhaps jsMath accomplishes this goal. In any case, LaTeX versions of the pages should be accessible to screenreaders. There may be other things we can do to improve accessibility, let's think more about that. The option of using really large fonts is one simple thing we could offer. 

\item "Talk page" for entries - Isn't this what the new "editorial areas" are supposed to provide? Or something similar, anyway? Another related idea would be to have world-editable versions of every entry so that people can just make changes and submit them as diffs (sort of like Scholarpedia), instead of having to write them out as formal "corrections". 

\item Stand-alone autolinker - NNexus. (It would be nice to know more about how far along things are...) 

\item Centralized Bibliographic Database - We have fielded data -- we should hurry up and get the database or whatever it's gonna be set up, so the data doesn't cease to be useful! 
\end{itemize}

\subsection*{Standing Request}

\begin{itemize}
\item version manager and mass uploading for content creators - Authors should be able to manage their submissions through an appropriate versioning system. We could even go one step easier and work with \url{http://getdropbox.com} (or something similar). A versioning system would also facilitate easy mass uploads (so needs to work with the new "tags" system). 

\item version manager read-only mode for downstream users - It should be possible to check any page or labeled collection of pages out of PlanetMath? with a simple command-line process. 

\item Redirect to login box after a 'you must be logged in' message - seems pretty obvious... 

\item A question-answering system - The initial request was for a "tip-based system like Google Answers": but even just keeping track of which forum postings are questions, which are answers, and which are just chit-chat would be helpful, and we could move on to more advanced things after that. (We absolutely need to keep track of which questions have answers and which ones don't! -- An interface that we can use to find the "unanswered questions" and also the "answered questions in such-and-such an area" would be great.) 

\item Message tagging by moderators - Extends the idea of keeping track of "questions" and "answers" to other types. Another useful feature would be to add bidirectional links to related encyclopedia topics! 

\item A version you can read on a cell phone - DVI with links? See this conference abstract: \url{http://www.tug.org/tug2009/abstracts/bazargan.txt} 

\item Forum UI improvements - "Threaded" version of the main forum feed; show the latest one or two messages in each forum, with the forum clearly marked; expand/collapse messages without leaving the message-index page; get access to past years and months quickly through a hierarchical menu (click on year to get list of months, click on month to bring up all messages); optional latex and/or ASCIImath rendering with jsMath; etc.! 

\item graph showing growth of encyclopedia - How has the encyclopedia changed and grown over time? It would be cool to see a graph, maybe sorted by MSC. 

\item shared style files - It could be convenient for everyone involved if we made a project to unify "style" (at least within MSCs): this would make notation more consistent and code more reusable. 

\item view tex source with links - This would be a variant on the standard non-linked tex source; e.g. when viewing \verb|\pmxref{ad hoc}{AdHoc?}|. "AdHoc?" would be a link to the object AdHoc?. 

\item List of rejected corrections - I think it would make sense to display "Latest Rejections" in the same way as "Latest Additions", "Latest Revisions" on the main page. This way, all users can verify that the rejection is valid. (Currently all corrections are viewable at \url{http://planetmath.org/?op=globalcors} -- but a variety of people have expressed dissatisfaction with the way rejected corrections are handled...) 

\item Other reporting on corrections - Show rejected corrections on a per-user basis; Show most-corrected articles and users; Show most-correcting users; etc. 

\item Incorporating 3rd party public domain content - There is a lot out there. I suggest talking to Ray about this, he has thought about it a lot. The most optimal thing would be to get a system installed for scanning, OCRing, uploading, subdividing and proofreading public domain mathematical writings into the PlanetMath? database... 

\item Completely expanded view of articles showing attachments, fora, corrections, etc - This would be useful. 

\item reject rejections - One way to approach the issue with corrections is to settle disputes by arbitration. E.g. if someone rejects a correction, and the corrector still disagrees, there could be a "settle dispute by arbitration" button that would call the matter to the attention of the content committee. 

\item syndication/subscriptions by keyword, metadata - The idea here is to enable people to specify certain keywords and phrases that they will get email notices about new postings in e.g. "number theory" or "prime numbers" or "38-XX". Similarly, one might wish to get email notices when certain people ("buddies") post. 

\item Centralized figure database/gallery - Currently figures are attached to a fixed entry. It would be good if all figures on PM could also be accessed in one central place. We could store the code for creating the figures here too. This will make it easier to learn how to make figures. 

\item hitcounters for messages - encyclopedia entries already have hitcounters; it should be easy to provide a similar counter supply such a counter for messages. 

\item Add comment when transfering entries - When transfering an entry to another user, it would make sense to have some comment field describing why one is offering the entry to him/her (like a bill of sale). 

\item Classification for requests - It would make sense to have the requests classified according to the MSC. Another useful way to add metadata to requests would be to add links to related entries. 

\item Converting forum discussions into PM entries - There is a large portion of math in the PM forums that could be turned into PM articles. We should be able to convert postings/discussions into articles! (Make it easy to do so...) 

\item User suite - In addition to the HTML profile... a place for personal files, user-owned tags to index favorite articles... basically we should rethink how PlanetMath? may be made more useful as a "social networking tool" 

\item Easily pending corrections to one's articles - This is just a basic usability thing! See \url{http://wiki.planetmath.org/AsteroidMeta/Personalized_pending_corrections} 

\item Entry copyright status - To help copyright screening and monitoring, it could be good to have one field for each entry reserved to record copyright notes. (See original suggestion at \url{http://wiki.planetmath.org/AsteroidMeta/Entry_copyright_field}) 

\item Retirement Home - Instead of deleted pages simply going away forever, they instead go to a retirement home. 

\item electronic map tackboard - It would be great to get a look at where users are for in-person networking! 

\item Math Of The Day - Like a word of the day service, but for math. Totally neat idea... it needs both human and perhaps a little system-level support. 

\item Data-type or subject-area-specific checklists - This would seem to require subject-area-level data (e.g. world-editable "introduction" pages for each MSC?) that would give appear and provide some guidelines or tips that are relevant when editing content in that area. 

\item automatically clickable links in requests, news - For some reason links aren't automatically made clickable for news and request items. 

\item Various small UI requests - \url{http://wiki.planetmath.org/AsteroidMeta/Noosphere_UI}

\item Endorsements - If one sees something amiss, one can file a correction. However, there is no way of pointing out what already is correct. The reason that this is problematic is that one cannot tell whether the absence of a correction should be taken to mean that an entry is correct or that it has not been examined. (We should revisit Pawel's "rankings" -- are people finding them useful? Were there other ideas that would be worth trying instead or in addition?) 

\item Better redirect after login - On PM, if I'm logged out, after I log in I get redirected to the main page. I'd strongly prefer to be redirected back to the page from which I logged in. 

\item Latest everything feed! - How about creating a page that lists in addition to the Latest Additions and Latest Revisions, the Latest Corrections, Latest Postings, and New Users, etc., all in one easy-to-access place? Then people will know where the action is! 

\item Cross index to the literature Knowing that a certain piece of information is to be found in a book can be of limited use because that might only narrow down the location to one of 500 pages. What PM-Xi would do would be to pinpoint the exact location so that one does not have to fumble around looking for needles in haystacks. 
\end{itemize}

\subsection*{On hold indefinitely!}

\begin{itemize}
\item improved task management features - At least for nonprofit administration, we have a really minimalistic "task manager" set up at \url{http://wiki.planetmath.org/AsteroidMeta/PlanetMath_Ongoing_Tasks} and this is good enough for the moment. At the same time, the idea of improving the workflow at PlanetMath? would be a reasonable thing to look at at some point.
\end{itemize}

\subsection*{Project Canceled!}

\begin{itemize}
\item nested noosphere instances - Although we aren't quite at the point of having "all of the Noosphere projects on one server", the way we're setting things up with tags may be "close enough". 
\end{itemize}
