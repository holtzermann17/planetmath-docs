\section*{I. Strengths}

\begin{paragraph}{Community and visibility}
I've estimated that PlanetMath users have contributed up
to \$0.75mil of writing.\footnote{\label{depreciation}That
  estimate was made about a year ago; the same instrument
  today suggests that our total value has
  \emph{depreciated} over the past year, to something more
  like \$0.6mil. See
  \url{http://metameso.org/~joe/docs/transparency.pdf} for
  details.}  During that same time frame, PlanetMath's
board members and major monetary donors have contributed
about \$0.25mil towards operational overhead.

In the same period, from 2005 to 2009, we've recieved an
estimated 30 million hits.  If each hit was valued at
\$0.50 (an arbitrary figure, chosen to match the value of
one Point in the estimates given above), then we've
delivered \$15 million worth of mathematical content, at a
cost of about \$1 million.

Apart from this back-of-the-envelope stuff, we're quite
popular: currently the \#2 hit on google for ``math
encyclopedia'' and \#5 hit on google for ``mathematics
encyclopedia''.  We show up on the first page of search
results for ``math''.
\end{paragraph}

\begin{paragraph}{Math-specific software}
PlanetMath works well with \LaTeX\ source code and has
state-of-the-art math-on-the-web rendering, making it an
attractive option for mathematicians.
\end{paragraph}

\begin{paragraph}{Content and coverage}
We currently have 8786 entries dealing with 15342
concepts.  While we don't have ``the most mathematics
content on the web'', without doubt, the content we have
is an asset.  For comparison's sake, there are
approximately 23488 mathematics articles on
Wikipedia\footnote{\url{http://en.wikipedia.org/wiki/Portal:Mathematics}}
and Springer claims to deal with more than 50000 concepts
in their online encyclopedia of
mathematics.\footnote{\url{http://eom.springer.de/}}
Still, PlanetMath tends to have a reputation for providing
in-depth treatment of topics that others may deal with
more superficially.  (It would be nice to back this up
with some solid evidence!)
\end{paragraph}

\begin{paragraph}{Free/Open}
Anyone can contribute to the project, and anyone can use
the things the project produces.
\end{paragraph}

\section*{II. Weaknesses}

\begin{paragraph}{No command-line or email-based access}
It may seem like a minor point, but for people who are
familiar with mainstream open source software development
patterns, the restriction to a web interface is a big
annoyance.  Lack of email integration in the forums leaves
us lightyears behind platforms like Google Groups or
Posterous in terms of useability and features.
\end{paragraph}

\begin{paragraph}{No access to the actual data}
PlanetMath's databases are not public.  There are
presumably many ways that the interested public could add
value to PlanetMath's data, if this data was available.
Even though we provide tarballs of content, our lack of
procedural openness compromises our extensibility and even
our legitimacy as an ``open'' project.
\end{paragraph}

\begin{paragraph}{No objective standards for quality}
Although a tool for rating articles was developed and
deployed on the production server, there is no easy way to
poll PlanetMath to find out more objectively useful things
like ``how many theorems are illustrated with examples''
or other simple structured queries.  This is a sign that
our data model may not be sufficiently rich.  (And, if
these questions are possible to ask in the backend, the
methods have not been exposed to the public.)
\end{paragraph}

\begin{paragraph}{No strategy for growth and adaptation, no public development agenda}
Despite the time-consuming and in some cases expensive
development of various strategic plans, PlanetMath
currently does not have a public roadmap or even, I'd
argue, any clearly stated and agreed-upon goals.  We do
have a Mission Statement, but that's
different.\footnote{\url{http://aux.planetmath.org/doc/mission.html}}
It seems our development agenda is so poorly maintained as
to be
non-existent.\footnote{\url{http://code.google.com/p/noosphere/wiki/RoadMap}}
\end{paragraph}

\begin{paragraph}{No protocol for review and update of site policies}
At a high, but still pragmatic level: we don't have any
protocols for changing basic things about the site (e.g.
things like the way it looks, the way it is divided into
sections, the scoring mechanism). Our policies such as
they are are currently managed in an \emph{ad hoc} fashion
in the community-moderated Site
Documents\footnote{\url{http://planetmath.org/?op=sitedoc}},
and on the PlanetMath-hosted AsteroidMeta
Wiki\footnote{\url{http://wiki.planetmath.org/}}.
\end{paragraph}

\begin{paragraph}{Slow-to-no transition from idea to implementation}
There have been about a million ideas developed over the
years, and contributed to the project via IRC, in the
forums, on the wiki, and at our rare in-person meetings --
but many of these ideas are now languishing.  Various
other related projects (e.g. SAGE), both large and small,
have successfully used ticket systems and so forth to sync
requests to development.  (Note: this critique of
PlanetMath's workflow covers nonprofit management, code
development \emph{and} math content development.)
\end{paragraph}

\begin{paragraph}{Money-poor}
When critical comments about development come up, we
remind ourselves that PlanetMath is almost ridiculously
poor in terms of financial resources.  Many of the people
involved with running the project are also quite short on
time these days (and as we know, time is money).
\end{paragraph}

\begin{paragraph}{Weak connections with related projects}
We're not very well connected to other math or software
projects.  Since we have no where near the amount of
contributor/developer effort as
e.g. SAGE\footnote{\url{http://www.sagemath.org/}} or
Elgg\footnote{\url{http://elgg.org/}} (or Wikipedia or
ArXiv for that matter), it would appear that by being
idiosyncratic, we may be losing out on potential
synergies.  Alternatively, we may just have to be even
more tenacious in our efforts to collaborate (e.g. to
combine with other established methods for peer review).
\end{paragraph}

\begin{paragraph}{``Why?''}
Despite the site's popularity, we have yet to learn just
\emph{why} people contribute to PlanetMath, or what they
are using it for.  If we could understand what people (or
institutions) are getting out of PlanetMath (as
contributors or users) -- and what they would \emph{like}
to get out of PlanetMath -- we could serve presumably
serve them better.  For example, we can say \emph{we're
  here to revolutionize education in Science, Technology,
  Engineering, and Mathematics} -- but for that to work,
well, we need to get it working.
\end{paragraph}

\section*{III. Opportunities}

\begin{paragraph}{We can be a leader in online math education and research support}
Despite the long list of problems enunciated above, one
else seems to be doing any better than PlanetMath at
providing an \emph{integrated} online mathematics help
tool.  I think we can still position ourselves at the head
of the online math pack -- if we choose to, and are
careful about how we do it.  We'll have to do something no
one else does, and at the same time, we'll want to work
well with everyone else too.  We can't just be a
``portal'' (other sites already do that), but I think it
is pretty clear that we can't just be an ``encyclopedia''
either.  If our idiosyncratic math-specific software is
one of our biggest strengths, we have to find ways to make
it work for us.  I think this will require us to fix at
least 50\% of the problems listed in the previous section.
\end{paragraph}

\begin{paragraph}{We can tap our existing friends and partners}
Although we have weak links to some of the important
players, we do have some important existing partnerships
(e.g. BKN, Springer) and some potential connections
(e.g. Design Science, KMi), as well as a few well-placed
friends (e.g. Jimmy Wales, Jim Pitman).  Assuming we're
capable of designing a ``PlanetMath'' that really works --
what do we then say to our pals to make it work better?
\end{paragraph}

\begin{paragraph}{Lots of content and tools available}
Not only is there all the stuff on Wikipedia (which we
could copy and edit, or even, presumably, snarf up from a
feed), there are tons of other free math and math-science
learning resources that use the exact same Creative
Commons license we use.  And there are constructive ways
to link to resources that don't use this license (and
maybe even generate a bit of cash flow in the process --
our ``Permanent Online Mathematics Pavillion'' idea).  If
PlanetMath is set up in a more extensible way, we'll be
able to use lots of other software \emph{tools} as well.
I think that extensibility is going to be key -- e.g. so
that we can finally build PlanetComputing and start
attracting some volunteer developers to the site.
\end{paragraph}

\section*{IV. Threats}

\begin{paragraph}{Stasis/Bitrot/Fog}
``Threats'' are supposed to come from outside -- but the
  biggest threat to PlanetMath seems to be stasis.  In
  fact, the site continues roughly as it has done for
  years -- but as Footnote \ref{depreciation} indicates,
  that's not going to be enough to preserve our value,
  much less grow and do more interesting things.  In fact,
  our software tools are apparently being completely
  renovated by James, but the status of this project is,
  at present, something of a mystery.  Without effective
  connections \emph{to} the outside, we will be forgotten,
  overshadowed, and ignored.  What's worse -- our great
  ideas \emph{still} won't be implemented.
\end{paragraph}

\begin{paragraph}{Newsflash: We \emph{are} the threat}
In a previous conversation with one professional society
about partnership possibilities, what we heard back was
that they were concerned that our publication model was a
threat to theirs.  Well, duh!  But the question shouldn't
be whose publication model will beat whose -- the question
is how to do the most good for mathematics.  We're only a
``threat'' to the standard publication model if we
\emph{produce content that's better, cheaper, and/or more
  accessible than the things others are producing}.
That's the best way forward for us to help mathematics,
too.  On the other hand, if we just produce a bunch of
crappy content, we're not doing anyone any good.  On that
note, I think this following quote shows the way forward:
\begin{quote}
The rapid growth of math resources on the web, which is
further pushed by wiki-based communities, is both a threat
and an opportunity for intelligent math learning
environments like ActiveMath.  On the one hand, the
continuing development of wiki content will soon dwarf
ActiveMath's carefully authored semantic-rich content; and
this will make its intelligent learning services that
exploit such limited but high-quality content less
enticing to use.  On the other hand, math learning
environments like ours could potentially profit from the
availability of math resources on the web.  If such
content were to be enriched with machine-readable
annotations that describe it, and if we could harness the
collaborative authoring process and encourage and guide
wiki authors to continually provide content and metadata,
then intelligent services could unleash their true
potential, with immediate return and added value for
authors and learners.  -- Claus Zinn \emph{Bootstrapping a
  Semantic Wiki Application for Learning Mathematics}
\end{quote}
\end{paragraph}
