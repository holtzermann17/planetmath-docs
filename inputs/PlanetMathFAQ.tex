\abstract{This based on the FAQ, written by mathwizard et
  all, which is available in its legacy format at
  \url{http://planetmath.org/?op=getobj&from=collab&id=35}.}

\subsection*{General Questions}
\subsubsection*{What is PlanetMath?}
PlanetMath is a free, collaborative, online mathematics encyclopedia. The stress is on peer review, rigour, openness, pedagogy, real-time content, interlinked content, and community-drivenness.

\subsubsection*{Is PlanetMath competing with MathWorld?}
Yes and No.
PlanetMath was started when MathWorld was taken offline, in order to comply with a CRC Press lawsuit against the author, Eric Weisstein. We simply assumed that there was no MathWorld anymore. We had become accustomed to using this type of resource, and simply put, nothing else could compare. The goal was then not to compete with MathWorld, but to make up for its absence with something that wasn't susceptible to the same pitfalls.

However, in order to generate a lot of content quickly and efficiently, it was obvious that collaboration was needed, so this made PlanetMath a horse of a different colour right from the start.

Now that MathWorld is back, you have two choices for finding this particular type of content (and MathWorld is significantly bigger), but we believe PlanetMath is more attractive to writers because of our open licensing (see \htmladdnormallink{GNU FDL}{http://www.gnu.org/copyleft/fdl.html}).

\subsubsection*{How do PlanetMath's goals differ from MathWorld's?}
Quantitatively, and in terms of mathematical coverage, there is no difference. Both PlanetMath and MathWorld have as a goal to be a comprehensive online encyclopedia of mathematics.

Qualitatively, there is a huge difference. PlanetMath is by nature built for collaborative authoring and peer-review. It is a ``bazaar'' instead of a ``cathedral'' (if you're into \htmladdnormallink{Raymondite}{http://catb.org/~esr/writings/cathedral-bazaar/} terminology). Posting happens on no one's authority but the user's. The guiding philosophy of the site's design is that the community can ``police itself.''

Another important goal of PlanetMath is to be immune from the courtroom, and to provide agreeable intellectual property rights to contributors. This is achieved through the \htmladdnormallink{GNU FDL}{http://www.gnu.org/copyleft/fdl.html} (Free Documentation License), which allows contributors to retain rights to their contributions, as well as PlanetMath.org (and others) to retain rights to the ``copy''.

We believe this ``free openness'' approach shifts the emphasis from a single instance of some content to the community of math-enthusiasts and themselves, and a sharing of their knowledge.

\subsubsection*{Is PlanetMath anything like Wikipedia (or Wiki software in general)?}
Yes. PlanetMath is similar to the Wiki model in that the content forms what is called a ``semantic network'' -- a set of interlinked concepts which derive meaning from both their content and their connections. The interlinking is of course exposed as HTML hyperlinks. As in semantic networks in general, the ``meaning'' of a node is not just its textual content, but also its position in the network. In this sense, PlanetMath and Wikipedia are a basis for ``holistic'' content and knowledge in general.

Where PlanetMath diverges is with its particular focus on TeX input and output, and serving the mathematics community. The correction system is completely apart from the Wiki model. Also, the base model of content authorship is that of object ``ownership'', where a single person acts as the gatekeeper to the object's content. However, it is also possible to adopt the Wiki model of universal editing on a per-entry basis, using PlanetMath's ACL system.

\subsubsection*{Is there a downloadable version of the PlanetMath encyclopedia?}
Yes, it is available both as a daily snapshot of the website, which contains the website as it is, and as a PDF file. The snapshot is updated at a daily basis and can be found \htmladdnormallink{here}{http://aux.planetmath.org/snapshots/}.

The whole content is also available as a large PDF file similar to a printable encyclopedia. This is available \htmladdnormallink{here}{http://aux.planetmath.org/book/}.

Also we may eventually try to get some support through companion print versions, although that is likely a long way off. We'd appreciate hearing from people on how to go about doing this.

\subsubsection*{Do I need to know \TeX/\LaTeX{} to write a PlanetMath entry?}
Yes. However, \LaTeX{} is not hard to learn, and we have purposely provided a clean way of viewing any existing PlanetMath entry in source form, to faciliate the learning of \TeX{} expression of mathematics.

There are a number of good online references and tutorials for TeX and LaTeX:
\begin{itemize}
\item \htmladdnormallink{Getting Started With \LaTeX}{http://www.maths.tcd.ie/~dwilkins/LaTeXPrimer/}
\item \htmladdnormallink{Hypertext Help with \LaTeX}{http://www.giss.nasa.gov/latex/}
\item \htmladdnormallink{The Harvard Guide to \TeX}{http://abel.math.harvard.edu/computing/latex/manual/texman.html}
\item \htmladdnormallink{A Guide to \LaTeX}{http://www.astro.rug.nl/~kuijken/latex.html}
\item \htmladdnormallink{The Not So Short Introduction to \LaTeX2e}{http://www.ctan.org/tex-archive/info/lshort/english/lshort.pdf}
\end{itemize}
Note that these are general guides. A PlanetMath-specific \LaTeX{} guide can can be found under ``Doc'' section in the main menu on the left.

\subsubsection*{Who can contribute?}
Anyone. There is no system of applying to be a contributor and subsequent granting of access: anyone can make an account and immediately start creating entries.

Of course, if you can't take constructive criticism, you might want to reconsider contributing.

\subsubsection*{Who reviews content, and how?}
Anyone who wants to. The chief means of peer review is via the corrections system. Anyone can submit corrections (of type addendum or erratum) to any entry. Points are given for accepted corrections, providing a means of crediting the reviewing party's contribution.

It is up to the author to determine of a correction is worth accepting. However, corrections are always available for viewing. If a correction is either wrongly accepted or rejected, others may notice this and file further corrections.

There is some ``authoritative review'' in the form of a small hand-picked set of users with editor permissions, who can make arbitrary changes to any entry. However, these editors are constrained by convention to only make grammatical or style edits, and they must provide comment to the entry owner for each edit. So ultimately their changes are still reversible, making the entry's owner the last word in all changes.

Currently there is no real recourse for someone who, say, writes poor entries and refuses all corrections, out of spite. In the future this will be handled by a ratings system, and filtering/sorting based on rating. Still, we have had no problem along these lines as of yet.

\subsubsection*{Who owns contributed material?}
Technically, the authors of content retain all rights to their work. However, in joining PlanetMath, all users agree to license to PlanetMath all content they add to the encyclopedia. Under this arrangement, authorship recognition must be preserved, but aside from that, anybody can make digital copies, print versions, or hard copy duplicates. Anybody can set up their own web site with content from PlanetMath. Anybody can sell compilations of that content, including (but not limited to) the authors.

For more details, see the \htmladdnormallink{GNU FDL}{http://www.gnu.org/copyleft/fdl.html}.

\subsubsection*{How can I help the project -- financially or otherwise?}
Please see our page on \htmladdnormallink{supporting PlanetMath}{http://aux.planetmath.org/doc/donate.html}.


\subsection*{Usage Questions}
\subsubsection*{How can I contribute?}
\begin{enumerate}
\item \htmladdnormallink{Make an account}{http://planetmath.org/?op=newuser}.
\item Browse around, get a feel for the site.
\item Read the New User \htmladdnormallink{documentation}{http://aux.planetmath.org/doc/}.
\item Log in, click on ``Encyclopedia'' under ``add'', in your user box.
\item Write your entry. Have a \LaTeX{} guide handy, if you don't already know \LaTeX{}.
\item Preview, submit.
\end{enumerate}

\subsubsection*{How do I know what content has already been added?}
You don't really need to, other than to avoid directly duplicating a concept someone else has already written an entry for. However, this can be avoided by simply using the search engine. In addition, the system will provide a warning at entry-adding time if the title of your entry is suspiciously similar to the title of an existing entry.

Aside from this, many people are worried about how they can possibly hyperlink their entries to others without knowing which things have been defined in PlanetMath already. Luckily you don't have to know this at all -- PlanetMath is designed to do the reference linking between entries automatically, and instantly.

This works both ways: not only will your entries immediately link to existing PlanetMath entries, but when new entries are added for concepts mentioned in your entry, your entry will actually be updated to link to them. The end result is that each person can be ignorant of the contents of the actual corpus (up to intent to add new entries.)

For more information about the automatic reference linking, see the expository document on \htmladdnormallink{Automatic Reference Linking}{http://planetmath.org/?op=getobj&from=collab&id=32} and the \htmladdnormallink{User Linking Controls}{http://planetmath.org/?op=getobj&from=collab&id=33} guide.

The reason this system works is that the particular content of PlanetMath is not novel -- it's all concepts that have already been thought of, that are well-known, and that are known by consistent handles. In general semantic net frameworks where new knowledge is formulated, there is less of a need for automatic linking, and this is less possible since concepts don't have well-known handles. Hence this isn't a big priority for ``brain-storming'' type Wiki-systems.

\subsubsection*{How do I view the \TeX{} source of an entry?}
Just set the ``View style'' (this control is below the display box for the entry) to ``TeX source''.

You can also view the preamble to the entry, by clicking on the ``view preamble'' link (this is also below the display box.) This is fairly important to to do to truly understand many entries, as authors typically define many of their own macros.

\subsubsection*{How do I include diagrams in my entry?}
Note that this question is \emph{not} the same as ``How do I include pictures in my entry?'' For our purposes, diagrams are logical figures, and are (or should be) scale and resolution-invariant.

This means that for diagrams, it is preferable to utilize a logical, description-based format, instead of a raster image (array of values representing pixels.)

For PlanetMath, the most useful of these formats are \Xy-pic and EPS (encapsulated postscript).

\Xy-pic is a language for describing figures directly within your TeX markup. This means you don't need to include separate files for images. \Xy-pic excels at simple geometric diagrams and array-based diagrams with arrows and lines of various styles between elements. You can see an example of \Xy-pic usage (and the source) on PlanetMath \htmladdnormallink{here}{http://planetmath.org/?op=getobj;from=objects;id=2865}.

EPS is a variant of the postscript language which is understood by \LaTeX. EPS files are separate from \TeX{} source and are included via the \texttt{\textbackslash{}includegraphics} directive. EPS is almost never generated by hand. Under unix, a good program for generating EPS files is \htmladdnormallink{xfig}{http://www.xfig.org/}. In windows, try \htmladdnormallink{Mayura Draw}{http://mayura.com} (both programs are free). \htmladdnormallink{Here}{http://planetmath.org/encyclopedia/BridgesOfKoenigsberg.html} and \htmladdnormallink{here}{http://planetmath.org/?op=getobj;from=objects;id=1452} are examples of figures generated by the EPS method in xfig and Mayura draw, respectively.

You can include EPS files by using the FileBox control (within the entry addition/editing form) to upload them.

\subsubsection*{I am having trouble reading an entry because the \TeX{} is not displaying correctly. What do I do?}
One thing that you can try doing is clearing your browser's cache, then reloading the page. If the problem persists, the entry most likely needs to be rerendered. You can rerender an entry as follows:

\begin{enumerate}
\item You need the id number of the entry. Just before the MSC of the entry appears, there is a line that begins \texttt{Object id is....}. This tells you the id number of the entry.
\item In the location bar of your browser, type in

\begin{center}
\begin{verbatim}
http://planetmath.org/?op=rerender&from=objects&id=xxxx
\end{verbatim}
\end{center}

with \texttt{xxxx} replaced with the id number of the entry. (Note that the id number need not be four digits long.) Once this is done, the page will rerender and appear in your browser.
\end{enumerate}

Note that this procedure does \emph{not} work on an entry that you are in the process of creating or editing. If the previewer is not working and you need to see the \TeX{} in an entry that you are creating or editing, then you need to save a compilable version of the entry.
