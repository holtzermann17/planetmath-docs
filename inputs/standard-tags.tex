\documentclass[12pt]{article}
\begin{document}

\title{Tags, Roles, and Actions for PlanetMath.org}
\author{by the PlanetMath Board of Directors}

\maketitle

\section{Summary}

We have a collection of private articles ({\bf NS:Private}), a
collection of public articles ({\bf NS:Public}), a collection of
submitted articles ({\bf NS:Submitted}), a collection of
editor-certified articles ({\bf NS:Approved}), and a collection
of editor-rejected articles ({\bf NS:Returned}).

All articles should be further categorized according to:

\begin{itemize}
\item {\bf \emph{Section}} - what basic sort of entry is it?
\item {\bf \emph{edlevel}} - how advanced is it?
\item {\bf \emph{expstyle}} - how is it written?
\item {\bf \emph{audience}} - for whom is it written?
\item {\bf \emph{application}} - what is the disposition or intent?
\item {\bf \emph{era}} - when is it from?
\item {\bf \emph{source}} - in what publication ``milieux'' would this content typically reside?
\item {\bf \emph{depth}} - how in-depth is the treatment here?
\item {\bf \emph{lifecycle}} - what authoring stage is the entry in?
\end{itemize}

Articles tagged {\bf NS:Approved}, {\bf
 Section:Reference}, and {\bf source:standard} would
correspond to what we've been calling ``the Encyclopedia''!

\subsection*{Process Notes}

\begin{enumerate}
\item To begin with the entire corpus should presumably be
 thrown into {\bf NS:Submitted} or perhaps just {\bf
 NS:Public}, fully categorized using our new set of
 tags, and then tagged as either as {\bf NS:Approved} or
 {\bf NS:Returned}!

\item Articles can be stripped of their {\bf NS:Approved}
 tag and tagged as {\bf NS:Returned} at any time.

\item When tags for {\bf NS:Approved} articles are changed
 by the author, editors should be notified.
\end{enumerate}


\section{BASIC WORKFLOW}

\subsection*{STEP 1}

Every new article should start out with one of these two
tags:

\begin{itemize}
\item[-] {\bf NS:Public} (anyone can see it when given a direct URL)
\item[-] {\bf NS:Private} (content is only accessible to
 Coauthors, or perhaps ``\emph{Approved Viewers}''?)
\end{itemize}

\noindent Authors can choose to move items between {\bf NS:Public} and
{\bf NS:Private} at will.

\subsection*{STEP 2}

Whenever the author thinks it is ready, it will be
submitted for review, and it will acquire the following
tag instead of its original tag. (``\emph{Approved
 Authors}'' can bypass this step and go directly from
Step 1 to Step 3.)

\begin{itemize}
\item[-] {\bf NS:Submitted} (Publication requested)
\end{itemize}

\noindent {\bf NS:Private} articles that are ``submitted'' should turn into
{\bf NS:Public} articles (after a warning).

\subsection*{STEP 3}

Case 1. If the editors find the submitted article to be
publication-worthy, the {\bf NS:Submitted} tag is deleted,
and the {\bf NS:Approved} tag is added. Approved articles
should have a full complement of tags for categorization
as specified in the following section. These tags can be
supplied by the author or editors.

\begin{itemize}
\item[-] {\bf NS:Approved} (Publication accepted)
\end{itemize}

\noindent Case 2. If the article is not deemed
publication-worthy, is moved out of {\bf NS:Submitted} (or
{\bf NS:Approved}) into {\bf NS:Public}, but it is also
given an {\bf NS:Returned} tag (which can be removed if
the article is resubmitted and approved):

\begin{itemize}
\item[-] {\bf NS:Returned} (Publication politely declined)
\end{itemize}

\section{TAGS FOR CATEGORIZATION}

These tags do not have a direct bearing on ``workflow'', but
rather, are intended to provide refined categorizations
that help make our resources more useful to downstream
users. The intention is that tags should be viewable by
category and by intersections and unions of categories.

\subsection*{SECTION}

\begin{itemize}
\item {\bf Section:Reference} - encyclopedia or handbook style
\item {\bf Section:Research} - original research or surveys, or its constituents
\item {\bf Section:Recreation} - fun stuff
\item {\bf Section:Education} - materials for learning and instruction
\end{itemize}

\subsection*{EDLEVEL}

\begin{itemize}
\item {\bf edlevel:elementary} - elementary school, 1-4 grades
\item {\bf edlevel:middleschool} - middle school, 5-8 grades
\item {\bf edlevel:highschool} - high school, 9-12 grades
\item {\bf edlevel:college} - introductory undergraduate
\item {\bf edlevel:mathmajor} - advanced undergraduate
\item {\bf edlevel:masters} - beginning graduate/prelim exams
\item {\bf edlevel:phd} - advanced graduate and professional
\end{itemize}

Note: when browsing, it may be helpful to have logical
categories ``school'', ``undergraduate'', and ``graduate'' that
draw material from the obvious tag collections.

\subsection*{EXPSTYLE}

\begin{itemize}
\item {\bf expstyle:popular} - simplified, impressionistic treatment of a topic
\item {\bf expstyle:pedagogical} - Intended for ``students'' -- rigorous, but with guidance.
\item {\bf expstyle:specialist} - no simplifying assumptions
\item {\bf expstyle:other} - non-standard forms of exposition
\end{itemize}

\subsection*{AUDIENCE}

\begin{itemize}
\item {\bf audience:student} - e.g. how to solve some textbook problems
\item {\bf audience:teacher} - e.g. how to illustrate a certain concept
\item {\bf audience:autodidact} - e.g. a tutorial
\item {\bf audience:researcher} - e.g. literature survey and open problems
\item {\bf audience:industry} - e.g. useful algorithms with examples
\end{itemize}

\subsection*{APPLICATION}

\begin{itemize}
\item {\bf application:pure} - Math for math's sake
\item {\bf application:applied} - Using math to solve problems outside of math
\item {\bf application:cultural} - Social, cultural, and historical context
\item {\bf application:interdisciplinary} - Stuff in between math and other disciplines
\end{itemize}

\subsection*{ERA}

\begin{itemize}
\item {\bf era:classic} - circa Euclid
\item {\bf era:modern} - circa Poincar\'e
\item {\bf era:contemporary} - circa the Internet
\end{itemize}

\subsection*{SOURCE}

\begin{itemize}
\item {\bf source:standard} - in text books
\item {\bf source:research} - in papers but not books
\item {\bf source:other} - in neither
\end{itemize}

\subsection*{DEPTH}

\begin{itemize}
\item {\bf depth:stub} - Just a starting point or even ``blank'' node with a name of a topic
\item {\bf depth:minimal} - States a fact with the minimum possible symbology and statements
\item {\bf depth:medium} - minimal plus some explanatory prose.
\item {\bf depth:maximal} - medium plus plentiful examples, demonstrations, and/or derivations; a complete treatment
\end{itemize}

\subsection*{LIFECYCLE}

\begin{itemize}
\item {\bf lifecycle:proto} - Just getting started.
\item {\bf lifecycle:evolving} - Rapidly developing content. Consumable, but may not be complete.
\item {\bf lifecycle:complete} - Fully-formed; fit for ``printing'', but still likely to go through many ``tuning'' revisions.
\item {\bf lifecycle:mature} - Content is complete but also ``stable'', with low need or desire for revisions.
\end{itemize}

\subsection*{FOR LATER...}

\begin{itemize}
\item {\bf Domain:Math}
\item {\bf Domain:Physics}
\item {\bf Domain:ComputerScience}
\item {\bf Domain:Biology}
\item {\bf Domain:Linguistics}
\end{itemize}

\end{document}

