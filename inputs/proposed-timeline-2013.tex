Each of the outcomes listed below ameliorates one risk or bottleneck.
The items here have been used to seed the issue tracker at
\url{https://github.com/holtzermann17/planetmath-docs/issues?state=open},
which will be kept up-to-date as we go forward.
\begin{description}
\item[Ongoing] \quad
\begin{itemize}
\item Answer questions, greet new members, and help integrate them into the community.
\item Make improvements to the software, especially being responsive to bugs pointed out by users.
\item {\bf Outcome}: show that there's a ``there, there.''
\end{itemize}
\item[Now: public beta testing (a month)] \quad 
\begin{itemize}
\item demonstrate how to use all of the existing features and debug as necessary
\item {\bf Outcome}: solidify our successes from the past 10 years
\end{itemize}
%
\item[Organization building to improve our "internal" documents (April/May)] \quad 
\begin{itemize}
\item Decide policies on which partners, license terms for content are acceptable and compatible with PM mission.
\item Decide what the future focus of our content building approach will be (at the level of licensing and other targets)
\item {\bf Outcome}: build a functional organizational model that we can fold subsequent policy documents into
\end{itemize}
%
\item[First round of Content building e.g. Calculus, College Algebra (June/July)] \quad 
\begin{itemize}
\item 2013 version of the FEM, PlanetMath edition of a few textbooks
\item {\bf Outcome}: Set up an editorial review system which will be generally be accepted by academia.
\end{itemize}
%
\item[Community building (August/September)] \quad 
\begin{itemize}
\item distribute flyers, mass mailing
\item post on other websites (math fora, social media, etc.)
\item inreach with the PlanetMath community (restart community discussions, start some new projects that contribute to community building, e.g. topic of the month, etc.)
\item Implement something like the 750words.com patronage system
\item Concrete proposals about the benefits of membership (we get some money and we can use it to address the aims of donors, give voting rights)
\item Additional Membership perks. (e.g.  allows you to express your appreciation, prominently published link to  your website, decoration on your username or whatever)
\item {\bf Outcome}: An active, interested, engaged community of contributors and a minimally functional business model
\end{itemize}
%
\item[Org-level Outreach, e.g. to Simon Foundation (October/November)] \quad
\begin{itemize}
\item plan a set of online services to complement existing offline offerings
\item draw up a detailed list of people who might want to invest in PlanetMath (e.g. sponsorship arrangement with a business, grant from a foundation, contract with an NGO), and establish contact
\item functional connection with other organizations (e.g. MAA members get free or reduced PlanetMath membership or whatever, or put in \$5 extra to also join PlanetMath, arrangement we had with Wikipedia w.r.t. fundraising campaign)
\item {\bf Outcome}: connecting with the mainstream of the math universe, e.g. journals, teachers, math reviews, publishers, etc.
\end{itemize}
%
\item[Build the software developer community (ongoing, reassess in December)] \quad 
\begin{itemize}
\item build at least some basic documentations about how to get involved (basics of the FTG platform)
\item {\bf Outcome}: Planetary is mostly self-sustaining
\end{itemize}
%
\item[Organization building (start in December 2013 and plan for 2014)] \quad 
\begin{itemize}
\item Membership drive
\item Newsletter
\item {\bf Outcome}: a more-than-just-minimally functional business model (e.g. like MAA, Universities, Sierra Club or whatever) with more money to put into enhancing the quality of our offerings (content, AI, functionality, etc.)
\end{itemize}
\end{description}
