\documentclass{article}

\usepackage[frenchlinks]{hyperref}

\setcounter{tocdepth}{1}
\usepackage{paralist}

\begin{document}

\title{Planetary System Metadata ``Rosetta Stone''}
\author{Joseph Corneli}
\date{\today}

\maketitle

\tableofcontents

\section*{Introduction}

This file serves as a guide to the data structures used in
the Planetary System.  Many of these structures correspond
to structures used on PlanetMath.org.  Complete
information on the legacy implementation of PlanetMath is
given in \cite{KrowneThesis}.  The legacy schema is summed
up at
\href{http://trac.mathweb.org/planetary/browser/pmredux-extras/pm-table-schema.txt?rev=HEAD}{pm-table-schema.txt}.

Work on a metadata translation started with articles; our
current bindings for articles are defined in
\href{http://trac.mathweb.org/planetary/browser/src/sty/planetmath-specials.sty.ltxml?rev=HEAD}{planetmath-specials.sty.ltxml}.
These bindings are reviewed below, along with numerous
other types of system objects.

Our strategy is to map metadata it is provided in the
original source format into Vanilla (or garden) database
tables, and into linked data forms suitable for publishing
as RDF or RDFa.  The original sources are typically
PlanetMath database tables, sometimes with a corresponding
user-editable source file.

For the Linked Data component, we draw on the following
ontologies (see also pp. 361--363 of \cite{LangeThesis}
for an index to an overview of these and other related
resources):

\bigskip

\begin{compactitem}
\item \href{http://www.w3.org/TR/rdf-concepts/}{RDF}: Resource Description Framework (RDF) \url{http://www.w3.org/1999/02/22-rdf-syntax-ns#}
\item \href{http://www.w3.org/TR/2004/REC-rdf-schema-20040210/}{RDFS}: RDF Vocabulary Description Language (``RDF Schema'') \url{http://www.w3.org/2000/01/rdf-schema#}
\item \href{http://www.w3.org/TR/2009/REC-owl2-overview-20091027/}{OWL}: Web Ontology Language (OWL) \url{http://www.w3.org/2002/07/owl#}
\item \href{http://dublincore.org/documents/dces/}{DCT}: Dublin Core \url{http://purl.org/dc/terms/}
\item \href{http://sioc-project.org/ontology}{SIOC}: Semantically-Interlinked Online Communities \url{http://rdfs.org/sioc/ns#}
\item \href{http://www.loc.gov/standards/mads/rdf/}{MADS} Metadata Authority Description Schema in RDF \url{http://www.loc.gov/mads/rdf/v1#}
\item \href{http://www.w3.org/TR/skos-reference/skos.html}{SKOS}: Simple Knowledge Organization System \url{http://www.w3.org/2004/02/skos/core#}
\item \href{http://kwarc.info/projects/docOnto/omdoc.html}{OMDoc}: OMDoc (Open Mathematical Documents) \url{http://omdoc.org/ontology#}
\item \href{http://xmlns.com/foaf/spec/}{FOAF}: Friend of a Friend \url{http://xmlns.com/foaf/0.1/}
\item \href{http://alpha.planetmath.org/spec/}{PM}: Our own \emph{ad hoc} vocabulary is turning into a proper ontology at \url{http://alpha.planetmath.org/pm/ns#}, however we will try to use existing standards instead where possible!
\item \href{http://alpha.planetmath.org/spec/}{VERSIONING}: tentative PlanetMath versioning vocabulary (\url{http://alpha.planetmath.org/pm/versioning#}), but we're trying to find a better one
\item \href{http://bblfish.net/work/atom-owl/2006-06-06/AtomOwl.html}{AWOL}: AtomOwl \url{http://bblfish.net/work/atom-owl/2006-06-06/#}
\item \href{http://moat-project.org/ontology}{MOAT}: Meaning of a Tag \url{http://moat-project.org/ns#}
\item \href{http://liris.cnrs.fr/Documents/Liris-4737-slides-sioca.pdf}{SIOCA}: \url{http://rdfs.org/sioc/actions#}
\item \href{http://motools.sourceforge.net/event/event.html}{EVENT}: \url{http://purl.org/NET/c4dm/event.owl#}

\end{compactitem}

\bigskip

Note: in what follows ``TBA'' stands for To-Be-Added or
To-Be-Addressed later -- maybe by someone other than me!

\section{Articles}


Note that unlike the other legacy objects, legacy articles
are drawn from PlanetMath's hand-rolled XML-formatted
version management system, not the database.

Articles have been reformatted to include a metadata block
like this:

\begin{verbatim}
\pmowner{jac}{4316}
\pmmodifier{jac}{4316}
\pmcreated{2008-03-20 20:20:20}
\pmmodified{2008-11-03 11:11:11}
\pmtitle{Made up article}
\pmtype{Definition}
\pmauthor{jac}{4316}
\pmclassification{msc}{03B05}
\pmclassification{msc}{03G05}
\pmsynonym{hocus pocus}{made up concept}
\pmrelated{Mathematosis}
\end{verbatim}
(Some additional information can be supplied but isn't
required.)  The information is currently mapped as
follows:

\begin{verbatim}
(the article itself) -> sioc:Item
  [or some more specific pm:Article]
\pmselfproof{}       -> pm:selfproof
\pmcomment{}         -> pm:comment

  [hopefully we can find something in a suitable
  changeset ontology (TBA); not rdfs:comment though!]

\pmdefines{}         -> pm:defines        (uri)

  [rdfs:subPropertyOf owl:inverseOf rdfs:isDefinedBy]

\pmrelated{}         -> skos:related      (uri)
\pmsynonym{}{}       -> skos:exactMatch   (uri)
                     skos:altLabel     (literal)

  [the altLabel can be inferred from the
  skos:prefLabel of the skos:exactMatch object]

\pmkeyword{}         -> dct:subject       (uri)

\pmparent[]{}        -> pm:parent         (uri)

  [rdfs:subPropertyOf owl:inverseOf sioc:attachment]

\pmowner{}{}         -> sioc:has_owner    (uri)
\pmmodifier{}{}      -> sioc:has_modifier (uri)

\pmrecord{}{}        -> dct:hasVersion
\pmcanonicalname     -> dct:identifier
\pmcreated           -> dct:created       (xsd:date)
\pmmodified          -> dct:modified      (xsd:date)
\pmtitle             -> dct:title
                     skos:prefLabel    (literal)

\pmauthor            -> sioc:has_creator       (uri)
  [for compatibility we might additionally create dct:creator and foaf:maker]
\pmclassification    -> dct:subject       (uri)

\pmtype              -> rdf:type

  [Existing types of PlanetMath articles are mapped like so:

  Axiom              -> omdoc:Axiom
  Conjecture         -> omdoc:Conjecture
  Corollary          -> omdoc:Corollary
  Definition         -> omdoc:Definition
  Example            -> omdoc:Example
  Proof              -> omdoc:Proof
  Derivation         -> omdoc:Proof
  Result             -> omdoc:Theorem
  Theorem            -> omdoc:Theorem
  Bibliography       -> omdoc:Bibliography

    [available in OMDoc]

  Algorithm          -> zomgdoc:Algorithm
  Data Structure     -> zomgdoc:DataStructure
  Problem            -> zomgdoc:Problem
  Solution           -> zomgdoc:Solution

    [should be in OMDoc; Problems and Solutions
    were not on the original version of PlanetMath]

  Application        -> omdoc:omtext
  Biography          -> omdoc:omtext
  Feature            -> omdoc:omtext

    [Specific kinds of nonmathematical text
    will be annotated with Tags, see that
    section.]

  Topic              -> pm:Topic

    [rdfs:subClassOf skos:Concept]]
\end{verbatim}

Note that some of the \emph{ad hoc} terms could be
replaced by elements of the SOIC vocabulary
(e.g. \verb|sioc:has_parent|, \verb|soic:has_modifier|).

%  ``changeset'' vocabulary, see
% \url{http://vocab.org/changeset/schema.html}.

The item that seems worth making a special comment about
here is {\tt synonym}, which is a triple relating two
terms (concepts).  The syntax used above doesn't make this
immediately clear.  The other triples relate the article
to its metadata in an obvious fashion.

The most recent version of each article will appear in
Vanilla as follows:

\begin{verbatim}
CanonicalName    -> GDN_Article:ArticleID
\pmauthor        -> GDN_Article:InsertUserID
\pmmodifier      -> GDN_Article:UpdateUserID
\pmtitle         -> GDN_Article:Title
\pmtitle         -> GDN_Article:TitleXHTML
\pmtitle         -> GDN_Article:Name
ArticleContents  -> GDN_Article:Body
ArticleFormatted -> GDN_Article:BodyXHTML
\pmcreated       -> GDN_Article:DateInserted
\pmmodified      -> GDN_Article:DateUpdated
\end{verbatim}

The database mapping defined in {\tt src/importeLegacy.pl}
is:

\begin{verbatim}
objects:uid        -> GDN_Article:ArticleID
objects:type       -> GDN_Article:ArticleTypeID
objects:userid     -> GDN_Article:InsertUserID
                  GDN_Article:UpdateUserID
objects:created    -> GDN_Article:DateInserted
objects:modified   -> GDN_Article:DateUpdated
objects:title      -> GDN_Article:Title
                      GDN_Article:TitleXHTML
objects:data       -> GDN_Article:Body
                      GDN_Article:BodyXHTML
objects:name       -> GDN_Article:Name

objects:parentid   -> nil
objects:preamble   -> nil
objects:related    -> nil
objects:synonyms   -> nil
objects:defines    -> nil
objects:keywords   -> nil
objects:hits       -> nil
objects:self       -> nil
objects:pronounce  -> nil
objects:version    -> nil
objects:linkpolicy -> nil
\end{verbatim}

\verb|CanonicalName|, \verb|ArticleContents|, and
\verb|ArticleFormatted| all need to be found, inferred, or
otherwise produced by whatever function is in charge of
caching articles in the database.

\subsection{Mathematical Subject Classification (access method)}

We made a quick (and not quite accurate conversion) of the
AMS MSC \cite{MSC2000}, which is used on PlanetMath as one
of the primary ways to access articles.  We will soon
switch to a more accurate conversion that comes from
\LaTeX ML.

In brief, the MSC translation uses SKOS terms like
\verb|hasTopConcept|, \verb|narrower|, \verb|Concept| to
depict the MSC's hierarchy.  We can then make SPARQL
queries like the following (which retrieves the parent
label for a given base concept):

\begin{verbatim}
PREFIX msc: <http://purl.org/ontology/mo/msc#>
PREFIX skos: <http://www.w3.org/2004/02/skos/core#>
 select $parent $label where {
   $parent skos:narrower msc:00A05;
      skos:prefLabel $label
\end{verbatim}

In addition to being directly useful, the MSC serves as a
proof of concept for future metadata-based access methods.

\subsection{Collaboration objects} \label{collab}

Collaboration objects are effectively just articles that
contain their own preambles.  Their database schema also
includes support for locks (which we will presumably
handle in a more generic way).  Effectively collaboration
objects are a first use case for tags; see Section
\ref{tags}.

\begin{verbatim}
\pmtag{collab}
\pmtag{published}
\pmtag{sitedoc}
\end{verbatim}

\section{Forum Posts}

Here we see an example of ``system-owned'' metadata vs
``user owned'' metadata.  For ease of communication, we
will pretend that this metadata is stored in
\LaTeX\ documents in a similar fashion to the above, but
at least in the legacy system this is definitely not the
case.  This way of writing is maintained throughout
subsequent sections.

\begin{verbatim}
\sysPostContext{CanonicalName}            % messages:objectid
\sysPostInReplyTo{msg28342138}            % messages:replyto
\sysPostThread{}                          % messages:threadid
\sysPostCreated{2002-12-19 01:34:11}      % messages:created
\sysPostBy{jac}                           % messages:userid
\pmPostSubject{Awesome idea!!}            % messages:subject
\pmPostBody{We should do this.}           % messages:body
\end{verbatim}

In terms of the metadata store, the mappings are as
follow:

\begin{verbatim}
(the post itself)   -> sioc:Post
\sysPostContext{}   -> sioc:has_container (sioc:Forum)
\sysPostInReplyTo{} -> sioc:reply_of
\sysPostThread{}    -> sioc:has_container (sioc:Thread)
\sysPostCreated{}   -> dct:created (xsd:date)
\sysPostBy{}        -> sioc:has_creator
\pmPostSubject{}    -> dct:title
\pmPostBody{}       -> awol:content
\end{verbatim}

This is how to model the body of a post.  We create explicit content objects that have MIME types.  One content object can refer to another one, from which it has been generated, e.g.\ XHTML+MathML could have been generated from a {\LaTeX} source.

\begin{verbatim}
@prefix :     <http://somewhere.in.planetary/.../database#> .
@prefix awol: <http://bblfish.net/work/atom-owl/2006-06-06/#> .
@prefix sioc: <http://rdfs.org/sioc/ns#> .
@prefix rdfs: <http://www.w3.org/2000/01/rdf-schema#> .
@prefix rdf:  <http://www.w3.org/1999/02/22-rdf-syntax-ns#> .
@prefix owl:  <http://www.w3.org/2002/07/owl#> .

:post123
  a sioc:Post;
  awol:content :text123 .

:text123
  a awol:Content;
  awol:type "application/xhtml+xml";
  awol:body "Hello <math><mi>x</mi></math>";
  awol:src :latex123 .

:latex123
  a awol:Content;
  awol:type "application/x-latex";
  awol:body "Hello $x$" .
\end{verbatim}

See also Section \ref{argumentation}.

There will be a representation in Vanilla as follows.
(This will be more clear when I've had a chance to look at
the relevant
\href{http://vanillaforums.org/addon/replyto-plugin}{threaded
  forum plugin}.)

\begin{verbatim}
messages:objectid   -> GDN_Discussion:DiscussionID

  [which we will have to retrieve from somewhere]

messages:objectname -> GDN_Article:Name
messages:replyto    -> GDN_Comment:ParentCommentID

  [If using the above-mentioned plugin.]

messages:threadid   -> GDN_Discussion:DiscussionID
messages:created    -> GDN_Discussion.DateUpdated
                       GDN_Discussion.DateLastComment

messages:userid     -> GDN_Comment:InsertUserID
messages:subject    -> GDN_Discussion:Name

  [But only if it is a new discussion.]

messages:body       -> GDN_Comment:Body
\end{verbatim}

\subsection{Personal Mail}

Personal mail is effectively a forum that anyone can post
to, but only one person can read.

\begin{verbatim}
\sysMailFrom{jac}                % mail:userfrom
\pmMailTo{rpuzio}                % mail:userto
\pmMailSubject{talk?}            % mail:subject
\pmMailBody{I'm free at 5 GMT.}  % mail:body
\pmMailSent{2001-08-25 18:34:09} % mail:sent
\sysMailRead{1}                  % mail:_read
\end{verbatim}

We will skip the Linked Data step for this material for
now (at the very least it depends on a good implementation
of permissions, see Section \ref{permissions}).

Vanilla implements private messaging with
``Conversations''.  Conversations are at least one step
better than PlanetMath insofar as they support multiple
recipients.  Several tables are used to provide support
for conversations, namely \verb|GDN_Conversation|,
\verb|GDN_UserConversation|, and
\verb|GDN_ConversationMessage|.

\begin{verbatim}
mail:userto   -> GDN_User:Contributors
                 GDN_Conversation:Contributors

  [GDN_Conversation has to be written in a special format,
  see below.]

mail:userfrom -> GDN_User:Contributors
                 GDN_Conversation:Contributors
                 GDN_ConversationMessage:InsertUserID

mail:subject  -> GDN_Conversation:Name                (add)

mail:body     -> GDN_ConversationMessage:Body
mail:sent     -> GDN_Conversation.DateInserted
                 GDN_ConversationMessage.DateInserted

mail:_read    -> GDN_User.CountUnreadConversations    (increment)

  [I think the application is broken; we can count all
  mail as unread for now I think, that's simplest.]
\end{verbatim}

The format of \verb|GDN_Conversation.Contributors| is
revealed as follows:

\begin{verbatim}
<?php
print_r(unserialize('a:2:{i:0;s:1:"1";i:1;s:1:"2";}
'));
?>

=>

Array
(
    [0] => 1
    [1] => 2
)
\end{verbatim}

where the array contents {\tt 1} and {\tt 2} are the IDs
of the participating users.  Since Noosphere only allows
two correspondents per thread, it will be easy to
translate the legacy data into the appropriate form.

\subsection{Notices}

Notices are effectively ``personal mail'' that originate
from the system itself.  Notices are primarily associated
with Watches (see Section \ref{watches}).

\begin{verbatim}
\sysNoticeTo{rpuzio}                        % notices:userid
\sysNoticeFrom{relatedBot}                  % notices:userfrom
\sysNoticeSubject{new "related" client bar} % notices:title
\sysNoticeCreated{2002-08-25 11:11:09}      % notices:created
\sysNoticeBody{Your article foo has been
              automatically linked to bar.} % notices:data
\sysNoticeRead{1}                           % notices:viewed
\sysNoticeChoice{Allow this?}               % notices:choice_title
\sysNoticeAction{AcceptRelatedLinkPrompt}   % notices:choice_action
\sysNoticeDefault{Allow}                    % notices:choice_default
\end{verbatim}

The last three elements, \verb|\sysNoticeChoice|,
\verb|\sysNoticeAction|, and \verb|\sysNoticeDefault|
would be good places to use the argumentation ontology
(Section \ref{argumentation}).  (Incidentally, it seems
these features are not in current use on PlanetMath.)

We will not do a linked data presentation at the moment,
and will move on to the Vanilla representation.

\begin{verbatim}
notices:userid         -> GDN_User:UserID
notices:userfrom       -> GDN_User:UserID
notices:title          -> GDN_Discussion:Name
notices:created        -> GDN_Discussion.DateUpdated
                          GDN_Discussion.DateLastComment
notices:viewed         -> GDN_Discussion.DateLastViewed (add)
notices:data           -> GDN_Comment:Body
notices:choice_title   -> GDN_Action.ActionName         (add)
notices:choice_action  -> GDN_Action.ActionID           (add)
notices:choice_default -> GDN_Action.ActionDefault      (add)
\end{verbatim}

\subsection{Legacy forums}

In PlanetMath, each article is a forum.  There are also
several special-purpose forums that aren't associated to
articles.  The contents of all of these need to be
imported into the new site.

\begin{verbatim}
\pmForumCreator{akrowne}                       % forums:userid
\pmForumCreated{2002-06-16 03:06:43}           % forums:created
\pmForumModified{2002-06-16 03:06:43}          % forums:modified
\pmForumParent{NULL}                           % forums:parentid
\pmForumTitle{PlanetMath Help}                 % forums:title
\pmForumDescription{This is the place
  to get assistance with tasks on PlanetMath.} % forums:data
\end{verbatim}

In terms of Linked Data:

\begin{verbatim}
\pmForumCreator{}     -> sioc:has_creator
\pmForumCreated{}     -> dct:created
\pmForumModified{}    -> dct:modified
\pmForumParent{}      -> sioc:has_parent
\pmForumTitle{}       -> dct:title
\pmForumDescription{} -> dct:description
\end{verbatim}

For Vanilla:

\begin{verbatim}
forums:userid   -> GDN_Discussion.InsertUserID
forums:created  -> GDN_Discussion.DateInserted
forums:modified -> GDN_Discussion.DateUpdated
forums:parentid -> GDN_Discussion.Parent        (add or ignore)
forums:title    -> GDN_Discussion.Name
forums:data     -> GDN_Discussion.Body
\end{verbatim}

Note that PlanetMath legacy forums currently do not make
use of the ``\verb|parent|'' field, so we could ignore it
for now.

\subsection{News}

Much like personal mail is a forum that anyone can post
to, but only one person can read, news is a forum that
everyone can read, but only a specific group of people
(site admins) can post to.  Presumably this is easy to set
permissions for in Vanilla (Section \ref{permissions}).

\begin{verbatim}
\pmNewsCreator{akrowne}                        % news:userid
\pmNewsCreated{2001-01-01 00:00:00}            % news:created
\pmNewsModified{2001-01-01 00:00:00}           % news:modified
\pmNewsTitle{The beginning}                    % news:title
\pmNewsIntro{Time stands still for a moment}   % news:intro
\pmNewsDescription{Let there be math!}         % news:body
\end{verbatim}

\begin{verbatim}
news:userid   -> GDN_Discussion.InsertUserID
news:created  -> GDN_Discussion.DateInserted
news:modified -> GDN_Discussion.DateUpdated
news:parentid -> GDN_Discussion.Parent        (add or ignore)
news:title    -> GDN_Discussion.Name
news:body     -> GDN_Discussion.Body
news:intro    -> GDN_Discussion.Summary       (add or merge into data)
\end{verbatim}

\begin{verbatim}
news:userid   -> sioc:has_creator
news:created  -> dct:created
news:modified -> dct:modified
news:title    -> sioc:has_parent
news:hits     -> dct:title
news:body     -> awol:content
news:intro    -> dct:description
  [for a more in-depth representation, we could consider awol:summary,
   whose range is awol:TextContent]
\end{verbatim}

\subsection{Argumentation support} \label{argumentation}

We plan to integrate Christoph Lange's
\href{https://svn.salzburgresearch.at/svn/kiwi/IkeWiki/branches/SWiM/trunk/WEB-INF/ontologies/matharg/matharg.owl}{argumentation
  ontology} into Planetary, and although it may in some
ways be more relevant to the Corrections section of this
document (Section \ref{corrections}), it is worth
mentioning here as well, since the idea of
``argumentation'' will make corrections into a more
discussion-oriented feature of the site than was
previously the case.

When we think about implementation, perhaps what works
best would be same set of tags attached to articles
(cf. Section \ref{tags}).  However, note that unlike
completely ``free-form'' tags, there would be specific
semantics attached to discourse markers from the
argumentation ontology.  (Other non-freeform tags might
have their own semantics attached as well, see the
relevant section for further discussion.)

\begin{verbatim}
\pmmarker{support}
\end{verbatim}

\section{User Homepage} \label{users}

Again, some user data is maintained by the system, like
``score''.  (Note that we'll eventually need a one or more
``analytics'' plugins to keep things like score up to date
-- we can start by taking a look at the
\href{http://vanillaforums.org/addon/topposters-plugin}{TopPosters
  plugin}.)

\begin{verbatim}
\pmUsername{jac}                            % users:username
\pmUserPassword{seeifyoucanguess}           % users:password
\pmUserEmail{holtzermann17@gmail.com}       % users:email
\pmUserForename{Joe}                        % users:forename
\pmUserSurname{Corneli}                     % users:surname
\pmUserCity{Milton Keynes}                  % users:city
\pmUserCounty{Buckinghamshire}              % users:state
% \pmUserState{}                            %
% \pmUserProvince{}                         %
\pmUserCountry{United Kingdom}              % users:country
\pmUserHomepage{http://metameso.org/~joe}   % users:homepage
\pmUserSignature{Something witty here}      % users:sig
\pmUserPrefs{need/to/check/what/this/means} % users:prefs
\pmUserBio{Bla bla bla.}                    % users:bio
\pmUserPreamble{\usepackage{amssymb}
                \usepackage{amsmath}}       % users:preamble
\sysUserScore{999}                          % users:score
\sysUserJoined{2002-12-19 01:34:11}         % users:joined
\sysUserAccessLevel{30}                     % users:access
\sysUserLastLogin{2010-03-03 13:08:13}      % users:last
\sysUserIsActive{1}                         % users:active
\sysUserLastIP{212.121.193.251}             % users:lastip
\sysUserKarma{0}                            % users:karma
\end{verbatim}

(We will aim to provide better internationalization of
addresses/locales in this implementation.)

In terms of the metadata store, the mappings are as
follow:

\begin{verbatim}
\pmUsername{}         -> sioc:UserAccount (uri)
\pmUserPassword{}     -> pm:password
\pmUserEmail{}        -> foaf:mbox
\pmUserForename{}     -> foaf:givenName
\pmUserSurname{}      -> foaf:familyName
\pmUserCity{}         -> mads:City
                         foaf:based_near
\pmUserCounty{}       -> mads:County
\pmUserState{}        -> mads:State
\pmUserProvince{}     -> mads:State
\pmUserCountry{}      -> mads:Country
\pmUserHomepage{}     -> foaf:homepage
\pmUserSignature{}    -> sig:signature    (maybe)
\pmUserPrefs{}        -> n/a
\pmUserBio{}          -> bio:Biography
\pmUserPreamble{}     -> pm:preamble      (awol:Content)
\sysUserScore{}       -> pm:score         (xsd:integer)
\sysUserJoined{}      -> pm:joined        (xsd:date)
\sysUserAccessLevel{} -> pm:accessLevel   (xsd:nonNegativeInteger)
\sysUserLastLogin{}   -> sioc:last_activity_date

  [Last activity is more granular than last login;
  Planetary should reflect the more granular data.
  If we want to keep track of login data, we can do
  that separately.]

\sysUserIsActive{}    -> pm:active         (xsd:boolean)
\sysUserLastIP{}      -> pm:lastIP
\sysUserKarma{}       -> pm:karma          (xsd:nonNegativeInteger)
  [maybe replace by sth. from a "trust" ontology]
\end{verbatim}

See see \url{http://xmlns.com/foaf/spec/} for info on FOAF
and \url{http://www.loc.gov/standards/mads/rdf/} for
details on madsrdf.  Another way to deal with country is
found in \url{http://vocab.org/reuters_regions/1.0/}.  Yet
another alternative is vcard,
\url{http://www.w3.org/Submission/vcard-rdf/}.

Note the idea of using digital signatures instead of
informal email signatures might be an inappropriate jump,
but see \url{http://payswarm.com/vocabs/signature}.  For
Bio stuff, see \url{http://vocab.org/bio/schema.rdf}

There will also be a representation in Vanilla as follows.

\begin{verbatim}
users:username -> GDN_User:UserID
users:password -> GDN_User:Password
users:email    -> GDN_User:Email
users:bio      -> GDN_User:About
users:joined   -> GDN_User:DateInserted
users:last     -> GDN_User:DateLastActive
users:prefs    -> GDN_User:Preferences    (maybe)
users:score    -> GDN_User:Score          (maybe)
users:access   -> GDN_User:Permissions    (maybe)
users:forename -> GDN_User:Forename       (add)
users:surname  -> GDN_User:Surname        (add)
users:city     -> GDN_User:City           (add)
users:state    -> GDN_User:State          (add)
users:country  -> GDN_User:Country        (add)
users:homepage -> GDN_User:Homepage       (add)
users:sig      -> GDN_User:Signature      (add)
users:preamble -> GDN_User:Preamble       (add)
users:active   -> GDN_User:ActiveUser     (add)
users:lastip   -> GDN_User:LastIP         (add)
users:karma    -> GDN_User:Karma          (add)
\end{verbatim}

Note that some of the things that need to be added might
be available in one of the existing plugins.

\section{Corrections} \label{corrections}

These are similar to forum posts in some regards, but they
have two ``phases''.  They are essentially like
article-level bug reports.  From a metadata point of view,
the vocabulary at \url{http://vocab.org/lifecycle/schema}
could prove useful.  Compare the diagrams on pp. 165 and
196 of \cite{KrowneThesis} for a picture of the way the
issue-tracking workflow at is envisioned.

\begin{verbatim}
\sysCorrectionContext{CanonicalName}       % corrections:objectid
\sysCorrectionCreated{2002-12-19 01:34:11} % corrections:filed
\sysCorrectionBy{jac}                      % corrections:userid
\pmCorrectionSubject{you missed a comma}   % corrections:title
\pmCorrectionBody{Check sentence two.}     % corrections:data
\pmCorrectionType{Minor}                   % corrections:type

\sysCorrectionClosed{2003-11-1 11:11:11}   % corrections:closed
\sysCorrectionClosedBy{apmxi}              % corrections:closedbyid
\sysCorrectionAccepted{1}                  % corrections:accepted
\pmCorrectionComment{Thanks}               % corrections:comment
\sysCorrectionGraceInterval{}              % corrections:graceint
\end{verbatim}

I don't yet know what the ``Correction Grace Interval''
is: will have to look more deeply.  Presumably it is a
reporter-specified grace period that is used to specify
the timeline in which the next automatic system action
takes place.

In terms of the metadata store, the mappings are as
follow:

\begin{verbatim}
\sysCorrectionContext{}       -> pm:parent                 (uri)
\sysCorrectionCreated{}       -> dct:created               (xsd:date)
\sysCorrectionBy{}            -> lifecycle:openedBy        (add)
\pmCorrectionSubject{}        -> lifecycle:subject         (add)
\pmCorrectionBody{}           -> lifecycle:taskDescription (add)
\pmCorrectionType{}           -> lifecycle:TaskGroup:meta  (check)

\sysCorrectionClosed{}        -> lifecycle:state:closed    (add)
\sysCorrectionClosedBy{}      -> lifecycle:closedBy        (add)
\sysCorrectionAccepted{}      -> lifecycle:completed
\pmCorrectionComment{}        -> lifecycle:closingComment  (add)
\sysCorrectionGraceInterval{} -> nil
\end{verbatim}

There will also be a representation in Vanilla as follows
(this requires creating an \verb|GDN_Issue| table.  (There
is an existing
\href{http://vanillaforums.org/addon/584-issue-tracker}{Issue
  Tracker} plugin, but it works by turning Discussions
into Issues, and that isn't what we want; however,
potentially some of the material could be recycled.)

\begin{verbatim}
corrections:objectid    -> GDN_Article:ArticleID
corrections:filed       -> GDN_Issue.DateUpdated
corrections:userid      -> GDN_Issue:InsertUserID
corrections:title       -> GDN_Issue:Name
corrections:data        -> GDN_Issue:Body
corrections:type        -> GDN_Issue:Type

corrections:closed      -> GDN_Issue:State
corrections:closedbyid  -> GDN_Issue:ClosedByID
corrections:accepted    -> GDN_Issue:Accepted
corrections:comment     -> GDN_Issue:ClosingComment
corrections:graceint    -> GDN_Issue:GracePeriod
\end{verbatim}

\section{Requests}
Similar to corrections; both have two phases and represent
some sort of ``bug reporting''.  In fact, we can reuse
much of the same infrastructure.

\begin{verbatim}
\sysRequestCreated{2002-12-19 01:34:11}  % requests:created
\sysRequestBy{jac}                       % requests:creatorid
\pmRequestSubject{what is a quasigroup?} % requests:title
\pmRequestBody{Just curious, thanks!}    % requests:data

\sysRequestClosed{2003-11-1 11:11:11}    % requests:closed
\sysRequestClosedBy{apmxi}               % requests:fulfillerid
\sysRequestFilled{1}                     % requests:fulfilled
\end{verbatim}

In terms of the metadata store, the mappings are as
follow:

\begin{verbatim}
\sysRequestCreated{}     -> dct:created (xsd:date)
\sysRequestBy{}          -> lifecycle:openedBy
\pmRequestSubject{}      -> lifecycle:subject
\pmRequestBody{}         -> lifecycle:taskDescription

\sysRequestClosed{}      -> lifecycle:state:closed
\sysRequestClosedBy{}    -> lifecycle:closedBy
\sysRequestFilled{}      -> lifecycle:completed
\end{verbatim}

In addition, we should make a
\verb|lifecycle:TaskGroup:request| annotation to the
object as a whole.

There will also be a representation in Vanilla as follows.

\begin{verbatim}
requests:created     -> GDN_Issue.DateUpdated
requests:creatorid   -> GDN_Issue:InsertUserID
requests:title       -> GDN_Issue:Name
requests:data        -> GDN_Issue:Body

requests:closed      -> GDN_Issue:State
requests:fulfillerid -> GDN_Issue:ClosedByID
requests:fulfilled   -> GDN_Issue:Accepted
\end{verbatim}

We might want to make an annotation for
\verb|GDN_Issue:Type| (set as request), and allow use of
\verb|GDN_Issue:ClosingComment| and
\verb|GDN_Issue:GracePeriod| similar to that for
Corrections.

Finally, note that although requests aren't currently made
in a given context, eventually they could be (as currently
happens with corrections).

\section{Groups}

It is possible to define groups within PlanetMath.  I can
imagine this functioning something like a group home-page.
For that to work, we'd likelywant to move the
\verb|group:description| text into a free-form section,
and make the group function like a special kind of object
(similar to forums).

\begin{verbatim}
\pmGroup{JCCoAuthors}{jac}{My pals.} % group:groupname
                                     % group:userid
                                     % group:description
\pmGroupMember{JCCoAuthors}{rpuzio}  % group_members:groupid
                                       (canonicalized)
                                     % group_members:userid
                                       (canonicalized)
\end{verbatim}

In terms of the metadata store, the mappings are as
follow.

\begin{verbatim}
\pmGroup{}{}{}          -> foaf:Group,
                           foaf:name,
                           foaf:maker,
                           foaf:homepage
\pmGroupMember{}{}      -> foaf:member
\end{verbatim}

There will also be a representation in Vanilla; we could use
the existing
\href{http://vanillaforums.org/addon/groups-plugin}{Groups}
and
\href{http://vanillaforums.org/addon/memberships-plugin}{Memberships}
plugins for this, though they would have to be tweaked in
order to (a) make it so that it is possible for
non-Administrators to create groups; and (b) so that
non-Administrators can edit group membership.  There are
other reasonable changes to make, like giving each group a
world-visible home page, and listing group membership on
user pages.

We should also consider adding access control settings for
groups (i.e. distinguish between groups that anyone can
join, or groups where control over membership is exclusive
to the group's administrator's, etc).

\begin{verbatim}
group:groupname        -> GDN_Group:Name
group:userid           -> GDN_Group:InsertUserID (add)
group:description      -> GDN_Group:Body         (add)
group_members:groupid  -> GDN_UserGroup:GroupID
group_members:userid   -> GDN_UserGroup:UserID
\end{verbatim}

\section{Tags} \label{tags}

We have some liberty about how we implement tags (and any
workflow that uses the tags), but it would be good to at
least consider the
\href{http://trac.mathweb.org/planetary/export/HEAD/contrib/planetmath-tags.pdf}{wishes}
of the PlanetMath board in this matter.  At the same time,
there are no legacy tags to import (though there are
certain data that can usefully be \emph{thought of} as
tags; some have already appeared in Section \ref{collab}).
A hypothetical use of tags (e.g. to describe a piece of
text that could be found in an introduction to Abstract
Algebra) would look like this:

\begin{verbatim}
\pmtag{Section:Reference,
       edlevel:mathmajor,
       expstyle:pedagogical,
       audience:student,
       application:pure,
       era:modern,
       source:standard,
       depth:medium,
       lifecycle:complete}
\end{verbatim}

We use the MOAT ontology (which itself relies on a
\href{http://www.holygoat.co.uk/owl/redwood/0.1/tags/}{tags}
ontology) to guide the Linked Data implementation.

\begin{verbatim}
\pmtag{} -> tags:name
            tags:taggedResource
            tags:taggedBy
            moat:tagMeaning
\end{verbatim}

The \href{http://vanillaforums.org/addon/tagging-plugin}{Tagging
  Enhanced} plugin gives a guide to the Vanilla
implementation.  (Update:

\section{Actions} \label{actions}

Activities are described with a couple of ontologies
related to the \href{http://activitystrea.ms}{activity
  streams} project, like
\href{http://xmlns.notu.be/aair/}{AAIR} or
\href{http://rdfs.org/sioc/actions}{SIOCA}.  We'll go with
SIOCA, so becoming familiar with
\href{http://liris.cnrs.fr/Documents/Liris-4573.pdf}{the paper}
that describes it could be a good idea.

The following pseudocode separates the system-level
\emph{definition} of a type of an ``action'' from the
record of a given \emph{instance} of an ``act''.

\begin{verbatim}
\sysAction{PostEvent}        % actions:type
          {EventSummary}     % actions:data
          {100}              % actions:score

\sysAct{UserName}            % actions:objectid
       {ForumName}           % actions:userid
       {PostEvent}
       {2001-01-01 00:00:00} % actions:created
\end{verbatim}

The Linked Data interpretation is like this:

\begin{verbatim}
actions:userid   -> sioc:has_creator
                    event:agent
actions:type     -> rdf:type
actions:objectid -> sioca:object
actions:data     -> dct:description
actions:created  -> dc:created
                    event:time
actions:score    -> pm:score
\end{verbatim}

The Vanilla interpretation is like this:

\begin{verbatim}
actions:userid   -> GDN_Activity.RegardingUserID
actions:type     -> GDN_Activity.ActivityTypeID
actions:objectid -> GDN_Activity.Route
actions:data     -> GDN_ActivityType.ProfileHeadline (with some interpretation)
actions:created  -> GDN_Activity.DateInserted
actions:score    -> GDN_ActivityType.Value           (add)
\end{verbatim}

\subsection{Watches} \label{watches}
A watch is a trigger or hook that responds to a given
action that occurs within the system.  In the PlanetMath
implementation the only thing that it watched for is a
generic ``update event''.

\begin{verbatim}
\pmWatch{JordansTotientFunction}{jac}  % watches:objectid, watches:tbl
                                       % watches:userid
\end{verbatim}

We will skip the Linked Data interpretation for now.

A (minimal) Vanilla interpretation is as follows.  (A more
complex version would allow the user to select the
functional behavior of the watch.)

\begin{verbatim}
watches:objectid -> GDN_Watch:TargetID
watches:tbl      -> GDN_Watch:TargetType
watches:userid   -> GDN_Watch:UserID
\end{verbatim}

\subsection{Hits}
Hits are in some sense ``anonymous actions'' coming from
outside the site.  It might be useful to have records that
pertain to hits queryable in the same style as records
about other user actions.

\section{Permission Policies} \label{permissions}

Permissions are managed in PlanetMath on a per-object
basis, typically by a per-object super-user, namely the
object's ``owner''.  This owner can also delegate similar
super-user rights to others (p. 146, \cite{KrowneThesis}).
Access control can also pass to groups (as defined in
previous section).

In Planetary, we will of course want to accommodate this
legacy style, but permissions will cover a more general
set of actions than before.  For example, a user may
typically have permission to update their own user
homepage, but they will not necessarily have permission to
adjust a field that says that they have solved a given
exercise (only another person, or a system process, would
be allowed to say that).

In what follows, \verb|x| corresponds to ``permission to
adjust the \verb|_acl| of an object'', in other words, the
only way to execute an object is to change the permissions
for that object.  In other applications (like the example
of getting credit for an exercise), execution may mean
something different.

\begin{verbatim}
\pmUserPermissions{ObjectName}{jac}{rwx}   % acl:objectid
                                           % acl:subjectid
                                           % acl:_read
                                           % acl:_write
                                           % acl:_acl

\pmDefaultPermissions{ObjectName}{r}       % as above, when
                                           % acl:default_or_normal
                                           %  = default

\pmGroupPermissions{ObjectName}{Group}{rw} % as above, when
                                           % acl:user_or_group
                                           %  = group
\end{verbatim}

A simple but widely applicable model of permissions would
indicate that permission applies to a subject, a verb, and
an object.  Our notation \begin{quotation}``\verb|{ObjectName}{jac}{rwx}|''
\end{quotation}
is in line with this understanding.  The generalization of
this notation would be
``\verb|{ObjectName}{UserName}{ActionName}|''.  However,
it may be that permission only applies in a given context,
so for example, we might use a string like
``\verb|{ModerationTag}{UserName}{x}{ForumName}|'' to say
that a given user has permission to apply a ``moderation''
tag in a particular forum, but not in every forum.  At
least in this case, we could use a ``reduced form''
and write \begin{quotation}
  ``\verb|{ForumName}{UserName}{ModeratorAction}|''.
\end{quotation}
Note that the SIOCA ontology allows several objects to be
used in one action (though it requires exactly one agent).

Another useful point to keep in mind is that system
permissions are associated with a global ``access level''
(cf. \verb|users:access|, which is currently just
translated to \verb|GDN_User:Permissions|, see Section
\ref{users}).

A full discussion of permissions will have to follow a
more complete understanding of user actions (Section
\ref{actions}).

For permission surrounding the use of tags, see
\href{http://trac.mathweb.org/planetary/export/HEAD/contrib/planetmath-tags.pdf}{this
  document} for the latest thoughts from the PlanetMath
board of directors.

There has been some discussion about how to integrate
permissions
\href{http://www.w3.org/2001/sw/Europe/events/foaf-galway/papers/fp/technical_and_privacy_challenges/}{into
  FOAF}, and a more recent discussion about how to make
the Semantic Web into an
``\href{http://www.freesoftwaremagazine.com/articles/semantic_web_operating_system_users_and_permissions}{operating
  system}''.  The most specific description of permissions
management for the read/write web I've been able to find
is
``\href{http://disi.unitn.it/security/spot.pdf}{Ontology
  Driven Community Access Control}'' or
``\href{http://eprints.biblio.unitn.it/archive/00001527/01/080.pdf}{RelBAC}''.
However, it seems a standard, public, ontology has not
been developed out of this work.

\begin{verbatim}
\pmUserPermissions{}{}{}  -> TBA
\pmDefaultPermissions{}{} -> TBA
\pmGroupPermissions{}{}{} -> TBA
\end{verbatim}

As for Vanilla, a new Access plugin that just does a
direct translation of the old PlanetMath ACL system could
be effective.

On the other hand, it would be nice to re-use Vanilla's
native ``Roles'' facility when and where possible.  For
example, we might be able to make a per-object dashboard
reusing some of the site-wide code from
\verb|/dashboard/user| (at least for user interface
purposes).

Note that in some sense ACLs are ``just additional article
metadata'' (like tags, so see Section \ref{tags}).

\begin{verbatim}
acl:objectid         -> GDN_Article:ArticleID
acl:subjectid        -> GDN_User:UserID,
                        GDN_Group:GroupID
                        (use "0" to denote default)
acl:_read            -> GDN_Access:Read         (add)
acl:_write           -> GDN_Access:Write        (add)
acl:_acl             -> GDN_Access:Administer   (add)
acl:user_or_group    -> GDN_Access:UserOrGroup  (add)
\end{verbatim}

\section{Links}

\emph{Note: We are going to defer automatic linking to
  Beta.  This section just contains some preliminary
  thoughts.}

Links are described in Chapter 9 of \cite{KrowneThesis}
(pp. 95-131), and an update of PlanetMath's automatic
linking system (``NNexus'') is described in \cite{NNexus}.
The code for NNexus is
\href{http://code.google.com/p/nnexus/source/checkout}{available},
along with documentation by James Gardner.

Quoting from the introduction to ``NNexus Installation'':

\begin{quotation}
\noindent When an entry is rendered for display (either at
display time or during offline batch processing), the text
is broken down into tokens and scanned for words that
invoke concepts that have been defined already (in other
entries).  These words (or word tuples) are ultimately
turned into hyperlinks to the corresponding entries in the
output rendering.  In addition, when the concepts are
added to the collection (or the set of concept labels
otherwise changes), entries containing potential
invocation of these concept labels are invalidated using a
special inverted index called the invalidation index.
This forces these entries to go through link analysis
themselves by or before the next time they are displayed.
\end{quotation}

Accordingly, working properly with links is directly
related to \emph{caching} and to PlanetMath's use of the
\verb|inv_dfs|, \verb|inv_idx|, and \verb|inv_phrases|
tables.

Taking a look at these tables we get a look at what's going on.

\begin{quote}
\begin{minipage}{5in}
\verb|inv_dfs|:
\begin{tabular}{|lll|}
\hline
{\bf id} &  {\bf word\_or\_phrase} & {\bf count} \\
\hline
99442 &              0 &   124 \\
99443 &              0 &   126 \\
99444 &              0 &   122 \\
\hline
\end{tabular}

\medskip

\verb|inv_idx|:
\begin{tabular}{|lll|}
\hline
{\bf id} &  {\bf word\_or\_phrase} & {\bf objectid} \\
\hline
99442 &              0 &   10 \\
99443 &              0 &   10 \\
99444 &              0 &   10 \\
\hline
\end{tabular}

\medskip

\verb|inv_phrases|:
\begin{tabular}{|ll|}
\hline
{\bf id} & {\bf id} \\
\hline
number inequality &     197463 \\
number degree     &     197464 \\
number constant   &     197465 \\
\hline
\end{tabular}
\end{minipage}
\end{quote}

\subsection{Semantic Links}

Links like this should be straightforward to add as
generalizations of tags, but we will save them until Beta
or later.  Furthermore, this sort of linking encroaches on
s\TeX\ territory, so let's reuse work from s\TeX\ here.

\begin{verbatim}
\pmILink{hasGeneralization}{SomeOtherArticle}
\pmOLink{SomeArticle}{hasGeneralization}{SomeOtherArticle}
\end{verbatim}

\appendix

\section{Legacy tables from PlanetMath.org}

These are the tables we have explicitly dealt with, or
replaced with some equivalent, above.

\begin{quotation}
\begin{minipage}{4in}
{\tt acl, actions, authors, classification, collab,
  concepts, corrections, forums, group\_members, groups,
  hits, links, mail, messages, msc, nag, news, notices,
  objindex, objlinks, requests, score, source, users,
  watches }
\end{minipage}
\end{quotation}

These are tables that we have not dealt with yet, but that
might be interesting to look at in the beta phase.
Generally they deal with features that were less important
than the core features described above.

\begin{quotation}
\begin{minipage}{4in}
{\tt blacklist, books, cache, catlinks, corstat, inv\_dfs,
  inv\_idx, inv\_phrases, inv\_words, ns, object\_rating,
  object\_rating\_all, ownerlog, papers, polls,
  relsuggest, rendered\_images, storage, tdesc,
  users\_clique, users\_rating, users\_uid\_seq }
\end{minipage}
\end{quotation}

These tables we may consider in passing, but they play a
supportive role and likely don't need to be imported
explicitly (though a few, like {\tt searchresults}, are of
possible historical interest).

\begin{quotation}
\begin{minipage}{4in}
{\tt acl\_default, acl\_default\_uid\_seq, acl\_uid\_seq,
  actions\_uid\_seq, blacklist\_uid\_seq, books\_uid\_seq,
  collab\_uid\_seq, concepts\_id\_seq,
  corrections\_uid\_seq, forums\_uid\_seq,
  groups\_groupid\_seq, lastmsg, lec\_uid\_seq,
  mail\_uid\_seq, messages\_tmp, messages\_uid\_seq,
  news\_uid\_seq, notices\_uid\_seq, objects\_uid\_seq,
  papers\_uid\_seq, polls\_uid\_seq, requests\_uid\_seq,
  searchresults }
\end{minipage}
\end{quotation}

\section{Tables in Vanilla}

These are the tables that Vanilla provides by default:

\begin{quotation}
\begin{minipage}{4in}
{\tt GDN\_Activity, GDN\_ActivityType, GDN\_AnalyticsLocal
  GDN\_Category, GDN\_Comment, GDN\_Conversation
  GDN\_ConversationMessage, GDN\_Discussion, GDN\_Draft
  GDN\_Invitation, GDN\_Message, GDN\_Permission,
  GDN\_Photo GDN\_Role, GDN\_Tag, GDN\_TagCategoryCount,
  GDN\_TagDiscussion GDN\_User, GDN\_UserAuthentication
  GDN\_UserAuthenticationNonce,
  GDN\_UserAuthenticationProvider
  GDN\_UserAuthenticationToken, GDN\_UserComment
  GDN\_UserConversation, GDN\_UserDiscussion,
  GDN\_UserMeta GDN\_UserRole }
\end{minipage}
\end{quotation}

We have modified many of these tables (as indicated in the
body of this document).  The new tables we introduced in
the process of adding PlanetMath content to Vanilla are as
follow:

\begin{quotation}
\begin{minipage}{4in}
{\tt GDN\_Access, GDN\_ActivityType, GDN\_Article, GDN\_ArticleType, GDN\_Group, GDN\_Issue, GDN\_UserGroup, GDN\_Watch }
\end{minipage}
\end{quotation}


\bibliography{pmlegacybib}{}
\bibliographystyle{annotations}

\end{document}
