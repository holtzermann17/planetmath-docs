\Xy-pic is a language for describing figures directly within your TeX markup. This means you don't need to include separate files for images. \Xy-pic excels at simple geometric diagrams and array-based diagrams with arrows and lines of various styles between elements. You can see an example of \Xy-pic usage (and the source) on PlanetMath \htmladdnormallink{here}{http://planetmath.org/?op=getobj;from=objects;id=2865}.

EPS is a variant of the postscript language which is understood by \LaTeX. EPS files are separate from \TeX{} source and are included via the \texttt{\textbackslash{}includegraphics} directive. EPS is almost never generated by hand. Under unix, a good program for generating EPS files is \htmladdnormallink{xfig}{http://www.xfig.org/}. In windows, try \htmladdnormallink{Mayura Draw}{http://mayura.com} (both programs are free). \htmladdnormallink{Here}{http://planetmath.org/encyclopedia/BridgesOfKoenigsberg.html} and \htmladdnormallink{here}{http://planetmath.org/?op=getobj;from=objects;id=1452} are examples of figures generated by the EPS method in xfig and Mayura draw, respectively.
